%%%%%%%%%%%%%%%%%%%%%%%%%%%%%%%%%%%%%%%%
%\begin{frame}
%\frametitle{Am I Taller Than The Average American Male? \hfill \includegraphics[scale = 0.05]{./images/clicker}}
%
%\framesubtitle{\href{http://www.cdc.gov/nchs/data/series/sr_11/sr11_252.pdf}{\fbox{Source: Centers for Disease Control (pg.\ 16)}}}
%
%My height is 73 inches. Based on a sample of US males aged 20 and over, the Centers for Disease Control (CDC) reported a mean height of about 69 inches in a recent report.
%
%\vspace{2em}
%
%
%
%\alert{Clearly I'm taller than the average American male!}\\
%Do you agree or disagree?
%\begin{enumerate}[(a)]
%	\item Agree
%	\item Disagree
%	\item Not Sure
%\end{enumerate}
%\end{frame}
%%%%%%%%%%%%%%%%%%%%%%%%%%%%%%%%%%%%%%%%%
%\section{Quick Review: Sampling Distribution of $\bar{X}_n$}
%%%%%%%%%%%%%%%%%%%%%%%%%%%%%%%%%%%%%%%%%
%\begin{frame}
%  \frametitle{What affects the accuracy of the sample mean?} 
%
%    If $X_1, \dots, X_n \sim \mbox{iid}$ with population mean $\mu$ and variance $\sigma^2$, then
%    \[ Var(\bar{X}_n) = Var\left( \frac{1}{n} \sum_{i=1}^n X_i \right)= \frac{\sigma^2}{n}\]
%
%\pause
%
%\begin{block}{Sample Size}
%  Smaller $n \implies$  easier to get an unrepresentative sample.
%\end{block}
%
%\pause
%
%\begin{block}{Population Variance}
% If everyone is very similar in height, we don't need a very large sample to estimate the population mean height accurately.
%\end{block}
%
%\end{frame}
%%%%%%%%%%%%%%%%%%%%%%%%%%%%%%%%%%%%%%%%%
%\begin{frame}
%\frametitle{Am I Taller Than The Average American Male?}
%\framesubtitle{\href{http://www.cdc.gov/nchs/data/series/sr_11/sr11_252.pdf}{\fbox{Source: Centers for Disease Control (pg.\ 16)}}}
%
%	
%	\begin{table}[h]
%	\caption{Height in inches for Males aged 20 and over (approximate)}
%		\begin{tabular}{|lr|}
%		\hline
%			Sample Mean & 69 inches\\
%			Sample Std.\ Dev.\ & 6 inches\\
%			Sample Size & 5647 \\
%			\hline
%			My Height & 73 inches\\
%			\hline
%		\end{tabular}
%	\end{table}
%
%\vspace{2em}
%\alert{We'll return to this example later\dots}
%
%\end{frame}
%%%%%%%%%%%%%%%%%%%%%%%%%%%%%%%%%%%%%%%%

\begin{frame}
  \frametitle{Today -- Simplest Example of a Confidence Interval}


\begin{itemize}
  \item Suppose the population is $N(\mu, \sigma^2)$
  \item We know $\sigma^2$ but not $\mu$
  \item Draw random sample $X_1, X_2, \hdots, X_n\sim \mbox{iid } N(\mu,\sigma^2)$ \pause
  \item Observe value of sample mean $\bar{x}_n$ (e.g.\ 69 inches) \pause
  \item What is a plausible range for $\mu$?\pause
  \item How confident are we? Can we make this precise?
\end{itemize}

\alert{Next time we'll look at more realistic and interesting examples\dots}

\end{frame}
%%%%%%%%%%%%%%%%%%%%%%%%%%%%%%%%%%%%%%%%

\begin{frame}
\frametitle{\includegraphics[scale = 0.05]{./images/clicker}}
Suppose $X_1, X_2, \hdots, X_n \sim \mbox{iid } N(\mu,\sigma^2)$. What is the sampling distribution of $\sqrt{n}(\bar{X}_n - \mu)/\sigma$?


\begin{enumerate}[(a)]
\item $N(\mu, \sigma^2)$
\item $N(0,1)$
\item $N(0,\sigma)$
\item $N(\mu, 1)$
\item Not enough information to determine.
\end{enumerate}

\end{frame}

%%%%%%%%%%%%%%%%%%%%%%%%%%%%%%%%%%%%%%%%
\begin{frame}
  \frametitle{$X_1, X_2, \hdots, X_n \sim \mbox{iid } N(\mu,\sigma^2)$}

	$$\sqrt{n}(\bar{X}_n - \mu)/\sigma =  \frac{\bar{X}_n - \mu}{\sigma/\sqrt{n}} =  \frac{\bar{X}_n - E[\bar{X}_n]}{SD(\bar{X}_n)}  \sim N(0,1)$$

  \vspace{1em}
Remember that we call the standard deviation of a sampling distribution the \alert{standard error}, written $SE$, so $$\frac{\bar{X}_n - \mu}{SE(\bar{X}_n)} \sim N(0,1)$$
\end{frame}
%%%%%%%%%%%%%%%%%%%%%%%%%%%%%%%%%%%%%%%%
%
%\begin{frame}
%\frametitle{\includegraphics[scale = 0.05]{./images/clicker}}
%Suppose $X_1, X_2, \hdots, X_n \sim \mbox{iid } N(\mu,\sigma^2)$. What is the approximate value of the following?
%$$P\left(-2 \leq \frac{\bar{X}_n - \mu}{SE(\bar{X}_n)} \leq 2 \right)\pause \alert{\approx 0.95}$$
%
%\end{frame}
%
%%%%%%%%%%%%%%%%%%%%%%%%%%%%%%%%%%%%%%%%%
\begin{frame}
\frametitle{What happens if I rearrange?}
	\begin{eqnarray*}
		P\left(-2 \leq \frac{\bar{X}_n - \mu}{SE(\bar{X}_n)} \leq 2 \right) &=& 0.95\\ \\ \pause
			P\left(-2\cdot SE\leq \bar{X}_n - \mu \leq 2 \cdot SE\right) &=& 0.95\\ \\ \pause
			P\left(-2\cdot SE - \bar{X}_n \leq - \mu \leq 2 \cdot SE - \bar{X}_n \right) &=& 0.95\\ \\ \pause
			\alert{P\left( \bar{X}_n - 2\cdot SE \leq \mu \leq \bar{X}_n + 2\cdot SE \right)}&\alert{=}& \alert{0.95}
	\end{eqnarray*}

\end{frame}

%%%%%%%%%%%%%%%%%%%%%%%%%%%%%%%%%%%%%%%%


\begin{frame}
\frametitle{Confidence Intervals}

\begin{block}{Confidence Interval (CI)}
Range $(A,B)$ constructed from the \alert{sample data} with specified probability of containing a \alert{population parameter}:
	$$P(A \leq \theta_0 \leq B) = 1-\alpha$$
\end{block} 

\pause

\begin{block}{Confidence Level}
The \alert{specified probability}, typically denoted $1-\alpha$, is called the confidence level. For example, if $\alpha = 0.05$ then the confidence level is 0.95 or 95\%.
\end{block}
\end{frame}
%%%%%%%%%%%%%%%%%%%%%%%%%%%%%%%%%%%%%%%%
\section{Confidence Interval for Mean of Normal Population ($\sigma^2$ Known)}
%%%%%%%%%%%%%%%%%%%%%%%%%%%%%%%%%%%%%%%%
\begin{frame}
\frametitle{Confidence Interval for Mean of Normal Population}
\framesubtitle{Population Variance Known}


	The interval \alert{$\boxed{\bar{X}_n \pm 2 \sigma/\sqrt{n}}$} has approximately 95\% probability of containing the population mean $\mu$, provided that:
		$$\boxed{X_1, X_2, \hdots, X_n\sim \mbox{iid } N(\mu,\sigma^2)}$$


\alert{But how are we supposed to interpret this?}

\end{frame}
%%%%%%%%%%%%%%%%%%%%%%%%%%%%%%%%%%%%%%%%
\section{Interpreting a Confidence Interval}
%%%%%%%%%%%%%%%%%%%%%%%%%%%%%%%%%%%%%%%%
%
%\begin{frame}
%\frametitle{Which quantities are random?\hfill \includegraphics[scale = 0.05]{./images/clicker}}
%Suppose $X_1, X_2, \hdots, X_n\sim \mbox{iid } N(\mu,\sigma^2)$.
%Which quantities are random variables?
%	\begin{enumerate}[(a)]
%\item $\mu$ only
%\item $\sigma$ and $\mu$
%\item $\sigma$ only
%\item $\sigma, \mu$ and $\bar{X}_n$
%\item $\bar{X}_n$ only
%\end{enumerate}
%
%\vspace{1em}
%\pause
%\alert{\large $\bar{X}_n$ only.}
%
%\end{frame}
%%%%%%%%%%%%%%%%%%%%%%%%%%%%%%%%%%%%%%%%
\begin{frame}
\frametitle{Confidence Interval is a Random Variable!}
\begin{enumerate}
	\item $X_1, \hdots, X_n$ are RVs $\Rightarrow \bar{X}_n$ is a RV (repeated sampling) \pause
	\item $\mu$, $\sigma$ and $n$ are constants \pause
	\item Confidence Interval $\bar{X}_n \pm 2 \sigma/\sqrt{n}$ is also a RV!
\end{enumerate}


\end{frame}

%%%%%%%%%%%%%%%%%%%%%%%%%%%%%%%%%%%%%%%
\begin{frame}
\frametitle{Meaning of Confidence Interval}
\begin{block}{Formal Meaning}
If we sampled many times we'd get many different sample means, each leading to a \alert{different} confidence interval. Approximately 95\% of these intervals will contain $\mu$.
\end{block}
 

\begin{block}{Rough Intuition}
What values of $\mu$ are consistent with the data?
\end{block}

\end{frame}




\begin{frame}
\frametitle{CI for Population Mean: Repeated Sampling}

\begin{center}
\setlength{\unitlength}{1cm}
\begin{picture}(5,7)
\put(-0.5,6){\framebox(6,1){$X_1, X_2, \hdots, X_n \sim \mbox{iid } N(\mu, \sigma^2)$}}

\pause

\put(0.5,6){\vector(-1,-1){1.5}}
\put(-2.3,3.7){\framebox(2.5,0.65){Sample 1}}

\pause

\put(-1,3.5){\vector(0,-1){0.75}}
\put(-1.25,2.1){\framebox(0.5,0.5){\small $\bar{x}_1$}}

\pause

\put(-1,2){\vector(0,-1){0.5}}
\put(-1.7,1){{\small $\bar{x}_1 \pm 2\sigma/\sqrt{n}$}}

\pause

\put(2,6){\vector(0,-1){1.5}}
\put(0.7,3.7){\framebox(2.5,0.65){Sample 2}}

\pause

\put(2,3.5){\vector(0,-1){0.75}}
\put(1.75,2.1){\framebox(0.5,0.5){\small $\bar{x}_2$}}

\pause

\put(2,2){\vector(0,-1){0.5}}
\put(1.35,1){{\small $\bar{x}_2 \pm 2\sigma/\sqrt{n}$}}

\pause

\put(3.8,4){\makebox{...}}
\put(3.8,2.3){\makebox{...}}

\pause

\put(4.5,6){\vector(1,-1){1.5}}
\put(4.8,3.7){\framebox(2.5,0.65){Sample M}}

\pause

\put(6,3.5){\vector(0,-1){0.75}}
\put(5.75,2.1){\framebox(0.5,0.5){\small $\bar{x}_M$}}

\pause

\put(6,2){\vector(0,-1){0.5}}
\put(5.35,1){{\small $\bar{x}_M \pm 2\sigma/\sqrt{n}$}}

\pause

\put(-1,0.2){\makebox{\small Repeat $M$ times $\rightarrow$  get $M$ different intervals}}

\pause

\put(-1,-0.3){\makebox{\small \alert{Large M $\Rightarrow$ Approx.\ 95\% of these Intervals Contain $\mu$}}}

\end{picture}
\end{center}


\end{frame}
%%%%%%%%%%%%%%%%%%%%%%%%%%%%%%%%%%%%%%%%
\begin{frame}
\frametitle{Simulation Example: $X_1, \hdots, X_5 \sim \mbox{iid } N(0,1)$, $M = 20$}

\begin{figure}
\centering
\includegraphics[scale = 0.5]{./images/CIs1}
\caption{Twenty confidence intervals of the form $\bar{X}_n \pm 2 \sigma/\sqrt{n}$ where $n=5$, $\sigma^2 = 1$ and the true population mean is $0$.}
\end{figure}

\end{frame}

%%%%%%%%%%%%%%%%%%%%%%%%%%%%%%%%%%%%%%%%
\begin{frame}
\frametitle{Meaning of Confidence Interval for $\theta_0$}
	$$\boxed{P(A\leq \theta_0 \leq B) = 1-\alpha}$$
Each time we sample we'll get a different confidence interval, corresponding to different realizations of the random variables $A$ and $B$. If we sample many times, approximately $100\times(1-\alpha)$\% of these intervals will contain the population parameter $\theta_0$.

\end{frame}





%%%%%%%%%%%%%%%%%%%%%%%%%%%%%%%%%%%%%%%%
%\begin{frame}
%\frametitle{True or False? \hfill \includegraphics[scale = 0.05]{./images/clicker}}
%Suppose 
%	$$\boxed{X_1, X_2, \hdots, X_n\sim \mbox{iid } N(\mu,\sigma^2)}$$
%Then the population mean $\mu$ has approximately a 95\% chance of falling in the interval $\bar{X}_n \pm 2 \sigma/\sqrt{n}$.
%
%\vspace{1em}
%
%\begin{enumerate}[(a)]
%\item True
%\item False
%\end{enumerate}
%
%
%\pause
%\vspace{1em}
%\alert{\huge FALSE! -- $\mu$  is a constant!}
%
%
%\end{frame}
%%%%%%%%%%%%%%%%%%%%%%%%%%%%%%%%%%%%%%%%%
\section{Margin of Error and Width}
%%%%%%%%%%%%%%%%%%%%%%%%%%%%%%%%%%%%%%%%%
\begin{frame}
\frametitle{Confidence Intervals: Some Terminology}
\begin{block}{Margin of Error}
When a CI takes the form $\widehat{\theta}\pm ME$, $ME$ is the Margin of Error. 
\end{block}
\pause
\begin{block}{Lower and Upper Confidence Limits}
The lower endpoint of a CI is the \alert{lower confidence limit (LCL)}, while the upper endpoint is the \alert{upper confidence limit (UCL)}.
\end{block}
\pause
\begin{block}{Width of a Confidence Interval}
The distance $\alert{|\mbox{UCL} - \mbox{LCL}|}$ is called the \alert{width} of a CI. This means exactly what it says. 
\end{block}

\end{frame}
%%%%%%%%%%%%%%%%%%%%%%%%%%%%%%%%%%%%%%%%
\begin{frame}
\frametitle{What is the Margin of Error\hfill \includegraphics[scale = 0.05]{./images/clicker}}
In the preceding example of a  95\% confidence interval for the mean of a normal population when the population variance is known, which of these is the \textbf{margin of error}?
	\begin{enumerate}[(a)]
		\item $\sigma/\sqrt{n}$
		\item $\bar{X}_n$
		\item $\sigma$
		\item $2\sigma/\sqrt{n}$
		\item $1/\sqrt{n}$
	\end{enumerate}
\pause
\vspace{1em}
\alert{\large $2\sigma/\sqrt{n}$, since the CI is $\bar{X}_n \pm 2\sigma/\sqrt{n}$}
\end{frame}
%%%%%%%%%%%%%%%%%%%%%%%%%%%%%%%%%%%%%%%%
\begin{frame}
\frametitle{What is the Width?\hfill \includegraphics[scale = 0.05]{./images/clicker}}
In the preceding example of a  95\% confidence interval for the mean of a normal population when the population variance is known, which of these is the \textbf{width} of the interval?
	\begin{enumerate}[(a)]
		\item $\sigma/\sqrt{n}$
		\item $2\sigma/\sqrt{n}$
		\item $3\sigma/\sqrt{n}$
		\item $4\sigma/\sqrt{n}$
		\item $5\sigma/\sqrt{n}$
	\end{enumerate}
\pause
\vspace{1em}
\alert{\large $4\sigma/\sqrt{n}$, since the CI is $\bar{X}_n \pm 2\sigma/\sqrt{n}$}
\end{frame}
%%%%%%%%%%%%%%%%%%%%%%%%%%%%%%%%%%%%%%%%
\begin{frame}
\frametitle{Example: Calculate the Margin of Error \hfill \includegraphics[scale = 0.05]{./images/clicker}}
\begin{center}\fbox{\begin{minipage}{0.75\textwidth}
$X_1, \hdots, X_{100} \sim \mbox{iid } N(\mu, 1)$ but we don't know $\mu$. Want to create a 95\% confidence interval for $\mu$.
\end{minipage}}
\end{center}
What is the margin of error?
\pause

\vspace{2em}

The confidence interval is $\bar{X}_n \pm 2\sigma/\sqrt{n}$ so 
	$$\alert{ME = 2\sigma/\sqrt{n} = 2 \cdot 1/\sqrt{100} = 2/10 = 0.2}$$

\end{frame}
%%%%%%%%%%%%%%%%%%%%%%%%%%%%%%%%%%%%%%%%
\begin{frame}
\frametitle{Example: Calculate the Lower Confidence Limit \hfill \includegraphics[scale = 0.05]{./images/clicker}}


\begin{center}\fbox{\begin{minipage}{0.75\textwidth}
$X_1, \hdots, X_{100} \sim N(\mu, 1)$ but we don't know $\mu$. Want to create a 95\% confidence interval for $\mu$.
\end{minipage}}
\end{center}

We found that $ME=0.2$. The sample mean $\bar{x} = 4.9$. What is the lower confidence limit?
\pause

\vspace{2em}

	$$\alert{\mbox{LCL} = \bar{x} - ME= 4.9 - 0.2 = 4.7}$$

\end{frame}
%%%%%%%%%%%%%%%%%%%%%%%%%%%%%%%%%%%%%%%%
\begin{frame}
  \frametitle{Example: Similarly for the Upper Confidence Limit\dots}


\begin{center}\fbox{\begin{minipage}{0.75\textwidth}
$X_1, \hdots, X_{100} \sim N(\mu, 1)$ but we don't know $\mu$. Want to create a 95\% confidence interval for $\mu$.
\end{minipage}}
\end{center}

We found that $ME=0.2$. The sample mean $\bar{x} = 4.9$. What is the upper confidence limit?
\pause

\vspace{2em}

	$$\alert{\mbox{UCL} = \bar{x} + ME= 4.9 + 0.2 = 5.1}$$

\end{frame}
%%%%%%%%%%%%%%%%%%%%%%%%%%%%%%%%%%%%%%%%
\begin{frame}
\frametitle{Example: 95\% CI for Normal Mean, Popn.\ Var.\ Known}

\begin{center}\fbox{\begin{minipage}{0.75\textwidth}
$X_1, \hdots, X_{100} \sim N(\mu, 1)$ but we don't know $\mu$.
\end{minipage}}
\end{center}

	$$\alert{\mbox{95\% CI for } \mu = [4.7, 5.1]}$$

What values of $\mu$ are plausible?

\pause
\vspace{1em}

\alert{The data actually came from a $N(5,1)$ Distribution.}

\end{frame}
%%%%%%%%%%%%%%%%%%%%%%%%%%%%%%%%%%%%%%%%
\begin{frame}
\frametitle{Want to be more certain? Use higher confidence level.}
What value of $c$ should we use to get a 100$\times(1-\alpha)$\% CI for $\mu$?
	\begin{eqnarray*}
		P\left(-c \leq \frac{\bar{X}_n-\mu}{\sigma/\sqrt{n}} \leq c \right) &=& 1-\alpha \\ \\ \pause
		P\left(\bar{X}_n - c \sigma/\sqrt{n} \leq \mu\leq \bar{X}_n + c \sigma/\sqrt{n} \right) &=& 1-\alpha 
	\end{eqnarray*}
 \pause

\alert{Take $c =$ \texttt{qnorm}$(1-\alpha/2)$} \pause
	$$\bar{X}_n \pm \texttt{qnorm}(1-\alpha/2) \times \sigma/\sqrt{n}$$
\end{frame}
%%%%%%%%%%%%%%%%%%%%%%%%%%%%%%%%%%%%%%%%
\begin{frame}
\begin{figure}
\centering
\includegraphics[scale = 0.6]{./images/normal_tails}
\end{figure}
\end{frame}
%%%%%%%%%%%%%%%%%%%%%%%%%%%%%%%%%%%%%%%%
%\begin{frame}
%\frametitle{Confidence Interval for a Normal Mean, $\sigma$ Known}
%\Large
%$$\boxed{\bar{X}_n \pm \texttt{qnorm}(1-\alpha/2) \times \sigma/\sqrt{n}}$$
%\end{frame}
%%%%%%%%%%%%%%%%%%%%%%%%%%%%%%%%%%%%%%%%%
\begin{frame}
\frametitle{What Affects the Margin of Error?}

	$$\boxed{\bar{X}_n \pm \texttt{qnorm}(1-\alpha/2) \times \sigma/\sqrt{n}}$$
 
  \small

	
\begin{block}{Sample Size $n$}
ME decreases with $n$: bigger sample $\implies$ tighter interval
\end{block}


\begin{block}{Population Std.\ Dev.\ $\sigma$}
ME increases with $\sigma$: more variable population $\implies$ wider interval
\end{block}



\begin{block}{Confidence Level $1-\alpha$}
ME increases with $1-\alpha$: higher conf.\ level $\implies$ wider interval

\pause

\vspace{1em}
	\begin{tabular}{r|lll}
	\hline
	Conf.\ Level & 90\% & 95\% & 99\% \\
	$\alpha$ & 0.1 & 0.05 & 0.01\\
	\texttt{qnorm}$(1-\alpha/2)$&1.64 & 1.96 & 2.56\\
	\hline
	\end{tabular}
\end{block}	
\end{frame}
%%%%%%%%%%%%%%%%%%%%%%%%%%%%%%%%%%%%%%%%
