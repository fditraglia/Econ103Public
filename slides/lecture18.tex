%%%%%%%%%%%%%%%%%%%%%%%%%%%%%%%%%%%%%%%%
\section{The Pepsi Challenge}
%%%%%%%%%%%%%%%%%%%%%%%%%%%%%%%%%%%%%%%%
\begin{frame}
\frametitle{The Pepsi Challenge}
Our expert claims to be able to tell the difference between Coke and Pepsi. Let's put this to the test! 
\begin{itemize}
\item Eight cups of soda 
	\begin{itemize}
\item Four contain Coke 
\item Four contain Pepsi 
\end{itemize}
	\item The cups are randomly arranged 
	\item How can we use this experiment to tell if our expert can \emph{\alert{really}} tell the difference?
\end{itemize}
\end{frame}
%%%%%%%%%%%%%%%%%%%%%%%%%%%%%%%%%%%%%%%%
\begin{frame}
\frametitle{The Results:}
	\# of Cokes Correctly Identified: \\ \vspace{2em}
	\alert{What do you think? Can our expert really tell the difference? \includegraphics[scale = 0.05]{./images/clicker}}
		\begin{enumerate}[(a)]
\item Yes
\item No
\end{enumerate}
\end{frame}

%%%%%%%%%%%%%%%%%%%%%%%%%%%%%%%%%%%%%%%%
\begin{frame}
\frametitle{\includegraphics[scale = 0.05]{./images/clicker}}
If you just guess randomly, what is the probability of identifying \emph{all four cups of Coke correctly}?
\pause
\begin{itemize}
\item ${8\choose 4}=70$ ways to choose four of the eight cups. \pause
\item If guessing randomly, each of these is \emph{\alert{equally likely}} \pause
\item Only \emph{\alert{one}} of the 70 possibilities corresponds to correctly identifying all four cups of Coke. \pause
\item Thus, the probability is $1/70 \approx 0.014$
\end{itemize}
\end{frame}
%%%%%%%%%%%%%%%%%%%%%%%%%%%%%%%%%%%%%%%%
\begin{frame}
\frametitle{Probabilities if Guessing Randomly}
	\begin{center}
		\begin{tabular}{rccccc}
		\hline
		\# Correct & 0 & 1 & 2 & 3 & 4\\
		Prob.&1/70 & 16/70 & 36/70 & 16/70 &1/70\\
		\hline
		\end{tabular}
	\end{center}
\end{frame}
%%%%%%%%%%%%%%%%%%%%%%%%%%%%%%%%%%%%%%%%
\begin{frame}
	\frametitle{\includegraphics[scale = 0.05]{./images/clicker}}
	\begin{center}
		\begin{tabular}{rccccc}
		\hline
		\# Correct & 0 & 1 & 2 & 3 & 4\\
		Prob.&1/70 & 16/70 & 36/70 & 16/70 &1/70\\
		\hline
		\end{tabular}
	\end{center}
	If you're just guessing, what is the probability of identifying \alert{\emph{at least}} three Cokes correctly?
	\pause
	\begin{itemize}
\item Probabilities of mutually exclusive events sum. 
\item $P$(all four correct) = 1/70 
\item $P$(exactly 3 correct )= 16/70 
\item $P$(at least three correct) $ = 17/70 \approx 0.24$

\end{itemize}
\end{frame}

%%%%%%%%%%%%%%%%%%%%%%%%%%%%%%%%%%%%%%%%
\begin{frame}
\frametitle{The Pepsi Challenge}
	\begin{itemize}
\item Even if you're just guessing randomly, the probability of correctly identifying three or more Cokes is around 24\%  \pause
\item In contrast, the probability of identifying \emph{\alert{all four}} Cokes correctly is only around 1.4\% if you're guessing randomly.  \pause
\item We should probably require the expert to get them all right\dots \pause 
\item What if the expert gets them all wrong? This also has probability $1.4\%$ if you're guessing randomly\dots \pause
\end{itemize}


\alert{That was a hypothesis test! We'll go through the details in a moment, but first an analogy\dots}
\end{frame}

%%%%%%%%%%%%%%%%%%%%%%%%%%%%%%%%%%%%%%%%
\section{Analogy between Hypothesis Testing and a Criminal Trial}
%%%%%%%%%%%%%%%%%%%%%%%%%%%%%%%%%%%%%%%%

\begin{frame}
%\frametitle{Hypothesis Testing -- Analogy to Criminal Trial}
\footnotesize
\begin{columns}
\begin{column}{6cm} 
 %FIRST COLUMN HERE
   	\begin{block}{Criminal Trial}
	\begin{itemize}
		\item<1-> The person on trial is either innocent or guilty (but not both!)
		\item<2-> ``Innocent Until Proven Guilty''
		\item<3-> Only convict if evidence is ``beyond a reasonable doubt''
		\item<4-> \emph{Not Guilty} rather than Innocent
			\begin{itemize}\footnotesize
				\item<5-> Acquit $\neq$ Innocent
			\end{itemize}
		\item<6-> Two Kinds of Errors:
			\begin{itemize} \footnotesize
				\item<6-> Convict the innocent
				\item<6->  Acquit the guilty
			\end{itemize}
		\item<7-> Convicting the innocent is a worse error. Want this to be rare even if it means acquitting the guilty.	\end{itemize}
\end{block}
   
   
\end{column} 
\begin{column}{6cm} 

 %SECOND COLUMN HERE 

\begin{block}{Hypothesis Testing}
		\begin{itemize}
		\item<1-> Either the null hypothesis $H_0$ or the alternative $H_1$  hypothesis is true.
		\item<2-> Assume $H_0$ to start
		\item<3-> Only reject $H_0$ in favor of $H_1$ if there is strong evidence.
		\item<4-> \emph{Fail to reject} rather than Accept $H_0$	
		\begin{itemize} \footnotesize
			\item<5-> (Fail to reject $H_0) \neq (H_0$ True) 
		\end{itemize}
				\item<6-> Two Kinds of Errors:
			\begin{itemize} \footnotesize
				\item<6-> Reject true $H_0$ (Type I)
				\item<6-> Don't reject false $H_0$ (Type II)
			\end{itemize}
			\item<7-> Type I errors (reject true $H_0$) are worse: make them rare even if that means more Type II errors.
	\end{itemize}
\end{block}

\end{column} 
\end{columns} 

\end{frame}
%%%%%%%%%%%%%%%%%%%%%%%%%%%%%%%%%%%%%%%%
\begin{frame}
\frametitle{How is the Pepsi Challenge a Hypothesis Test?}
	\begin{block}{Null Hypothesis $H_0$}
		Can't tell the difference between Coke and Pepsi: just guessing. \pause
\end{block}
	\begin{block}{Alternative Hypothesis $H_1$}
	Able to tell which ones are Coke and which are Pepsi.\pause
\end{block}
	\begin{block}{Type I Error -- Reject $H_0$ even though it's true} 
	Decide expert can tell the difference when she's really just guessing. \pause
\end{block}
	\begin{block}{Type II Error -- Fail to reject $H_0$ even though it's false}
	Decide expert just guessing when she really can tell the difference. 
\end{block}
\end{frame}
%%%%%%%%%%%%%%%%%%%%%%%%%%%%%%%%%%%%%%%%
\section{Steps in a Hypothesis Test}
%%%%%%%%%%%%%%%%%%%%%%%%%%%%%%%%%%%%%%%%
\begin{frame}
  \frametitle{How do we carry out a hypothesis test?}

  \begin{block}{Step 1 -- Specify $H_0$ and $H_1$}
   \begin{itemize}
     \item Pepsi Challenge: $H_0$ -- our ``expert'' is guessing randomly \pause
     \item Pepsi Challenge: $H_1$ -- our ``expert'' can tell which is Coke
   \end{itemize}
  \end{block}

  \pause

  \begin{block}{Step 2 -- Choose a Test Statistic $T_n$}
    \begin{itemize}
      \item $T_n$ uses sample data to measure the plausibility of $H_0$ vs.\ $H_1$ \pause
      \item Pepsi Challenge: $T_n =$ Number of Cokes correctly identified \pause
      \item Lots of Cokes correct $\Rightarrow$ implausible that you're just guessing
    \end{itemize}
  \end{block}


\end{frame}

%%%%%%%%%%%%%%%%%%%%%%%%%%%%%%%%%%%%%%%%
\begin{frame}

  \frametitle{Step 3 -- Calculate Distribution of $T_n$ under $H_0$}

    \begin{itemize}
      \item \alert{Under the null = Under $H_0$ = Assuming $H_0$ is true} \pause
      \item To carry out our test, need sampling dist.\ of $T_n$ under $H_0$ \pause
      \item $H_0$ must be ``specific enough'' that we can do the calculation. 
      \item Pepsi Challenge:
        \vspace{1em}
            \begin{center}
              \begin{tabular}{rccccc}
                \hline
                \# Correct & 0 & 1 & 2 & 3 & 4\\
                Prob.&1/70 & 16/70 & 36/70 & 16/70 &1/70\\
                \hline
              \end{tabular}
            \end{center}
    \end{itemize}
    
  
\end{frame}
%%%%%%%%%%%%%%%%%%%%%%%%%%%%%%%%%%%%%%%%
\begin{frame}
  \frametitle{Step 4 -- Choose a Critical Value $c$}

    \small
            \begin{center}
              \begin{tabular}{rccccc}
                \hline
                \# Correct & 0 & 1 & 2 & 3 & 4\\
                Prob.&1/70 & 16/70 & 36/70 & 16/70 &1/70\\
                \hline
              \end{tabular}
            \end{center}

      \begin{itemize}
      \item Pepsi Challenge: correctly identify many cokes $\Rightarrow$ implausible you're guessing at random. \pause
        \item Decision Rule: reject $H_0$ if $T_n > c$, where \alert{$c$ is the critical value}. \pause
        \item Choose $c$ to ensure $P(\text{Type I Error})$ is small. But how small? \pause
        \item \alert{Significance level $\alpha$} = max.\ prob.\ of Type I error we will allow \pause
      \item Choose $c$ so that if $H_0$ is true $P(T_n > c) \leq \alpha$ \pause 
    \item Pepsi Challenge: if you are guessing randomly, then
        \begin{itemize}
          \item $P(T_n > 3) = 1/70 \approx 0.014$
          \item $P(T_n > 2) = 16/70 + 1/70 \approx 0.23$
        \end{itemize}
      \end{itemize}

\end{frame}
%%%%%%%%%%%%%%%%%%%%%%%%%%%%%%%%%%%%%%%%
\begin{frame}
  \frametitle{How do we carry out a hypothesis test?}
            \begin{center}
              \begin{tabular}{rccccc}
                \hline
                \# Correct & 0 & 1 & 2 & 3 & 4\\
                Prob.&1/70 & 16/70 & 36/70 & 16/70 &1/70\\
                \hline
              \end{tabular}
            \end{center}
  \begin{enumerate}
    \item[Step 1] -- Specify Null Hypothesis $H_0$ and alternative Hypothesis $H_1$
    \item[Step 2] -- Choose Test Statistic $T_n$ 
    \item[Step 3] -- Calculate sampling dist of $T_n$ under $H_0$
    \item[Step 4] -- Choose Critical Value $c$
    \item[Step 5] -- Look at the data: if $T_n >c$, reject $H_0$.
  \end{enumerate}

  \begin{alertblock}{Pepsi Challenge}
    If $\alpha = 0.05$ we need $c = 3$ so that $P(T_n >3) \leq \alpha$ under $H_0$.
    Based on the results for our expert, would we reject $H_0$?
  \end{alertblock}
\end{frame}
%%%%%%%%%%%%%%%%%%%%%%%%%%%%%%%%%%%%%%%%
