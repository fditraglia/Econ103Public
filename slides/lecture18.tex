%%%%%%%%%%%%%%%%%%%%%%%%%%%%%%%%%%%%%%%%
\section{The Pepsi Challenge}
%%%%%%%%%%%%%%%%%%%%%%%%%%%%%%%%%%%%%%%%
\begin{frame}
\frametitle{The Pepsi Challenge}
Our expert claims to be able to tell the difference between Coke and Pepsi. Let's put this to the test! 
\begin{itemize}
\item Eight cups of soda 
	\begin{itemize}
\item Four contain Coke 
\item Four contain Pepsi 
\end{itemize}
	\item The cups are randomly arranged 
	\item How can we use this experiment to tell if our expert can \emph{\alert{really}} tell the difference?
\end{itemize}
\end{frame}
%%%%%%%%%%%%%%%%%%%%%%%%%%%%%%%%%%%%%%%%
\begin{frame}
\frametitle{The Results:}
	\# of Cokes Correctly Identified: \\ \vspace{2em}
	\alert{What do you think? Can our expert really tell the difference? \includegraphics[scale = 0.05]{./images/clicker}}
		\begin{enumerate}[(a)]
\item Yes
\item No
\end{enumerate}
\end{frame}

%%%%%%%%%%%%%%%%%%%%%%%%%%%%%%%%%%%%%%%%
\begin{frame}
\frametitle{\includegraphics[scale = 0.05]{./images/clicker}}
If you just guess randomly, what is the probability of identifying \emph{all four cups of Coke correctly}?
\pause
\begin{itemize}
\item ${8\choose 4}=70$ ways to choose four of the eight cups. \pause
\item If guessing randomly, each of these is \emph{\alert{equally likely}} \pause
\item Only \emph{\alert{one}} of the 70 possibilities corresponds to correctly identifying all four cups of Coke. \pause
\item Thus, the probability is $1/70 \approx 0.014$
\end{itemize}
\end{frame}
%%%%%%%%%%%%%%%%%%%%%%%%%%%%%%%%%%%%%%%%
\begin{frame}
\frametitle{Probabilities if Guessing Randomly}
	\begin{center}
		\begin{tabular}{rccccc}
		\hline
		\# Correct & 0 & 1 & 2 & 3 & 4\\
		Prob.&1/70 & 16/70 & 36/70 & 16/70 &1/70\\
		\hline
		\end{tabular}
	\end{center}
\end{frame}
%%%%%%%%%%%%%%%%%%%%%%%%%%%%%%%%%%%%%%%%
\begin{frame}
	\frametitle{\includegraphics[scale = 0.05]{./images/clicker}}
	\begin{center}
		\begin{tabular}{rccccc}
		\hline
		\# Correct & 0 & 1 & 2 & 3 & 4\\
		Prob.&1/70 & 16/70 & 36/70 & 16/70 &1/70\\
		\hline
		\end{tabular}
	\end{center}
	If you're just guessing, what is the probability of identifying \alert{\emph{at least}} three Cokes correctly?
	\pause
	\begin{itemize}
\item Probabilities of mutually exclusive events sum. 
\item $P$(all four correct) = 1/70 
\item $P$(exactly 3 correct )= 16/70 
\item $P$(at least three correct) $ = 17/70 \approx 0.24$

\end{itemize}
\end{frame}

%%%%%%%%%%%%%%%%%%%%%%%%%%%%%%%%%%%%%%%%
\begin{frame}
\frametitle{The Pepsi Challenge}
	\begin{itemize}
\item Even if you're just guessing randomly, the probability of correctly identifying three or more Cokes is around 24\%  \pause
\item In contrast, the probability of identifying \emph{\alert{all four}} Cokes correctly is only around 1.4\% if you're guessing randomly.  \pause
\item We should probably require the expert to get them all right\dots \pause 
\item What if the expert gets them all wrong? This also has probability $1.4\%$ if you're guessing randomly\dots \pause
\end{itemize}


\alert{That was a hypothesis test! We'll go through the details in a moment, but first an analogy\dots}
\end{frame}

%%%%%%%%%%%%%%%%%%%%%%%%%%%%%%%%%%%%%%%%
\section{Analogy between Hypothesis Testing and a Criminal Trial}
%%%%%%%%%%%%%%%%%%%%%%%%%%%%%%%%%%%%%%%%

\begin{frame}
%\frametitle{Hypothesis Testing -- Analogy to Criminal Trial}
\footnotesize
\begin{columns}
\begin{column}{6cm} 
 %FIRST COLUMN HERE
   	\begin{block}{Criminal Trial}
	\begin{itemize}
		\item<1-> The person on trial is either innocent or guilty (but not both!)
		\item<2-> ``Innocent Until Proven Guilty''
		\item<3-> Only convict if evidence is ``beyond a reasonable doubt''
		\item<4-> \emph{Not Guilty} rather than Innocent
			\begin{itemize}\footnotesize
				\item<5-> Acquit $\neq$ Innocent
			\end{itemize}
		\item<6-> Two Kinds of Errors:
			\begin{itemize} \footnotesize
				\item<6-> Convict the innocent
				\item<6->  Acquit the guilty
			\end{itemize}
		\item<7-> Convicting the innocent is a worse error. Want this to be rare even if it means acquitting the guilty.	\end{itemize}
\end{block}
   
   
\end{column} 
\begin{column}{6cm} 

 %SECOND COLUMN HERE 

\begin{block}{Hypothesis Testing}
		\begin{itemize}
		\item<1-> Either the null hypothesis $H_0$ or the alternative $H_1$  hypothesis is true.
		\item<2-> Assume $H_0$ to start
		\item<3-> Only reject $H_0$ in favor of $H_1$ if there is strong evidence.
		\item<4-> \emph{Fail to reject} rather than Accept $H_0$	
		\begin{itemize} \footnotesize
			\item<5-> (Fail to reject $H_0) \neq (H_0$ True) 
		\end{itemize}
				\item<6-> Two Kinds of Errors:
			\begin{itemize} \footnotesize
				\item<6-> Reject true $H_0$ (Type I)
				\item<6-> Don't reject false $H_0$ (Type II)
			\end{itemize}
			\item<7-> Type I errors (reject true $H_0$) are worse: make them rare even if that means more Type II errors.
	\end{itemize}
\end{block}

\end{column} 
\end{columns} 

\end{frame}
%%%%%%%%%%%%%%%%%%%%%%%%%%%%%%%%%%%%%%%%
\begin{frame}
\frametitle{How is the Pepsi Challenge a Hypothesis Test?}
	\begin{block}{Null Hypothesis $H_0$}
		Can't tell the difference between Coke and Pepsi: just guessing. \pause
\end{block}
	\begin{block}{Alternative Hypothesis $H_1$}
	Able to tell which ones are Coke and which are Pepsi.\pause
\end{block}
	\begin{block}{Type I Error -- Reject $H_0$ even though it's true} 
	Decide expert can tell the difference when she's really just guessing. \pause
\end{block}
	\begin{block}{Type II Error -- Fail to reject $H_0$ even though it's false}
	Decide expert just guessing when she really can tell the difference. 
\end{block}
\end{frame}
%%%%%%%%%%%%%%%%%%%%%%%%%%%%%%%%%%%%%%%%
\section{Steps in a Hypothesis Test}
%%%%%%%%%%%%%%%%%%%%%%%%%%%%%%%%%%%%%%%%
\begin{frame}
  \frametitle{How do we carry out a hypothesis test?}

  \begin{block}{Step 1 -- Specify $H_0$ and $H_1$}
   \begin{itemize}
     \item Pepsi Challenge: $H_0$ -- our ``expert'' is guessing randomly \pause
     \item Pepsi Challenge: $H_1$ -- our ``expert'' can tell which is Coke
   \end{itemize}
  \end{block}

  \pause

  \begin{block}{Step 2 -- Choose a Test Statistic $T_n$}
    \begin{itemize}
      \item $T_n$ uses sample data to measure the plausibility of $H_0$ vs.\ $H_1$ \pause
      \item Pepsi Challenge: $T_n =$ Number of Cokes correctly identified \pause
      \item Lots of Cokes correct $\Rightarrow$ implausible that you're just guessing
    \end{itemize}
  \end{block}


\end{frame}

%%%%%%%%%%%%%%%%%%%%%%%%%%%%%%%%%%%%%%%%
\begin{frame}

  \frametitle{Step 3 -- Calculate Distribution of $T_n$ under $H_0$}

    \begin{itemize}
      \item \alert{Under the null = Under $H_0$ = Assuming $H_0$ is true} \pause
      \item To carry out our test, need sampling dist.\ of $T_n$ under $H_0$ \pause
      \item $H_0$ must be ``specific enough'' that we can do the calculation. 
      \item Pepsi Challenge:
        \vspace{1em}
            \begin{center}
              \begin{tabular}{rccccc}
                \hline
                \# Correct & 0 & 1 & 2 & 3 & 4\\
                Prob.&1/70 & 16/70 & 36/70 & 16/70 &1/70\\
                \hline
              \end{tabular}
            \end{center}
    \end{itemize}
    
  
\end{frame}
%%%%%%%%%%%%%%%%%%%%%%%%%%%%%%%%%%%%%%%%
\begin{frame}
  \frametitle{Step 4 -- Choose a Critical Value $c$}

    \small
            \begin{center}
              \begin{tabular}{rccccc}
                \hline
                \# Correct & 0 & 1 & 2 & 3 & 4\\
                Prob.&1/70 & 16/70 & 36/70 & 16/70 &1/70\\
                \hline
              \end{tabular}
            \end{center}

      \begin{itemize}
      \item Pepsi Challenge: correctly identify many cokes $\Rightarrow$ implausible you're guessing at random. \pause
        \item Decision Rule: reject $H_0$ if $T_n > c$, where \alert{$c$ is the critical value}. \pause
        \item Choose $c$ to ensure $P(\text{Type I Error})$ is small. But how small? \pause
        \item \alert{Significance level $\alpha$} = max.\ prob.\ of Type I error we will allow \pause
      \item Choose $c$ so that if $H_0$ is true $P(T_n > c) \leq \alpha$ \pause 
    \item Pepsi Challenge: if you are guessing randomly, then
        \begin{itemize}
          \item $P(T_n > 3) = 1/70 \approx 0.014$
          \item $P(T_n > 2) = 16/70 + 1/70 \approx 0.23$
        \end{itemize}
      \end{itemize}

\end{frame}
%%%%%%%%%%%%%%%%%%%%%%%%%%%%%%%%%%%%%%%%
\begin{frame}
  \frametitle{How do we carry out a hypothesis test?}
            \begin{center}
              \begin{tabular}{rccccc}
                \hline
                \# Correct & 0 & 1 & 2 & 3 & 4\\
                Prob.&1/70 & 16/70 & 36/70 & 16/70 &1/70\\
                \hline
              \end{tabular}
            \end{center}
  \begin{enumerate}
    \item[Step 1] -- Specify Null Hypothesis $H_0$ and alternative Hypothesis $H_1$
    \item[Step 2] -- Choose Test Statistic $T_n$ 
    \item[Step 3] -- Calculate sampling dist of $T_n$ under $H_0$
    \item[Step 4] -- Choose Critical Value $c$
    \item[Step 5] -- Look at the data: if $T_n >c$, reject $H_0$.
  \end{enumerate}

  \begin{alertblock}{Pepsi Challenge}
    If $\alpha = 0.05$ we need $c = 3$ so that $P(T_n >3) \leq \alpha$ under $H_0$.
    Based on the results for our expert, would we reject $H_0$?
  \end{alertblock}
\end{frame}
%%%%%%%%%%%%%%%%%%%%%%%%%%%%%%%%%%%%%%%%
\section{Hypothesis test for the mean of a normal population}
%%%%%%%%%%%%%%%%%%%%%%%%%%%%%%%%%%%%%%%%
\begin{frame}
  \frametitle{Another Simple Example}
  
  Suppose $X_1, \dots, X_{100} \sim \mbox{ iid N}(\mu, \sigma^2 = 9)$ and we want to test
  \[
    \begin{array}{c}
      H_0\colon \mu = 2\\
      H_1\colon \mu \neq 2\\
    \end{array}
  \]

  \pause

  \begin{enumerate}
    \item[Step 1] -- Specify Null Hypothesis $H_0$ and alternative Hypothesis $H_1$ $\textcolor{blue}{\checkmark}$\pause
    \item[Step 2] -- \alert{Choose Test Statistic $T_n$}
  \end{enumerate}

  \pause

  If $\bar{X}$ is far from 2 then $\mu=2$ is implausible. Why?

\end{frame}
%%%%%%%%%%%%%%%%%%%%%%%%%%%%%%%%%%%%%%%%
\begin{frame}
  \frametitle{If $\bar{X}_n$ is far from 2, then $\mu = 2$ is implausible}

  Since $X_1, \dots, X_{100} \sim \mbox{ iid N}(\mu, 9)$, \alert{if $\mu = 2$ then $\bar{X} \sim N(2, 0.09)$}
  \begin{eqnarray*}
     P(a \leq \bar{X} \leq b) &=& \pause P\left(\frac{a - 2}{3/10} \leq \frac{\bar{X}-2}{3/10} \leq \frac{b - 2}{3/10} \right)\\ \pause
     &=& P\left( \frac{a-2}{0.3} \leq Z \leq \frac{b-2}{0.3} \right)
  \end{eqnarray*}
  where $Z \sim N(0,1)$ so we see that if $H_0\colon \mu = 2$ is true then \pause
  \begin{eqnarray*}
    P(1.7 \leq \bar{X} \leq 2.3) &=& P(-1 \leq Z \leq 1) \approx 0.68\\ \pause
    P(1.4 \leq \bar{X} \leq 2.6) &=& P(-2 \leq Z \leq 2) \approx 0.95 \\ \pause
    P(1.1 \leq \bar{X} \leq 2.9) &=& P(-3 \leq Z \leq 3) > 0.99 
  \end{eqnarray*}

\end{frame}
%%%%%%%%%%%%%%%%%%%%%%%%%%%%%%%%%%%%%%%%
\begin{frame}
  \frametitle{Step 2 -- Choose Test Statistic $T_n$}
  \begin{itemize}
    \item Reject $H_0\colon \mu = 2$ if the sample mean is far from $2$. \pause
    \item $\Rightarrow T_n$ should depend on the \alert{distance} from $\bar{X}$ to 2, i.e.\ $|\bar{X} - 2|$.\pause
    \item We can make our subsequent calculations much easier if we choose a \alert{scale for $T_n$ that is convenient under $H_0$\dots}
  \end{itemize}
  \begin{eqnarray*}
    \mu=2 \Rightarrow \quad \bar{X} - 2 &\sim& N(0, 0.09) \\ \\ \pause
    \frac{\bar{X} - 2}{0.3} &\sim& N(0,1)
  \end{eqnarray*}

  \alert{So we will set $\displaystyle T_n = \left|\frac{\bar{X} - 2}{0.3}\right|$}

\end{frame}
%%%%%%%%%%%%%%%%%%%%%%%%%%%%%%%%%%%%%%%%
\begin{frame}
  \frametitle{Another Simple Example: $X_1, \dots, X_{100}\sim \mbox{iid N}(\mu, \sigma^2 = 9)$}
  \begin{enumerate}
    \item[Step 1] -- $H_0\colon \mu = 2, \; H_1\colon \mu \neq 2$ $\textcolor{blue}{\checkmark}$
    \item[Step 2] -- $T_n = \displaystyle \left|\frac{\bar{X} - 2}{0.3} \right|$ $\textcolor{blue}{\checkmark}$
    \item[Step 3] -- If $\mu = 2$ then $\displaystyle \left(\frac{\bar{X} - 2}{0.3}\right) \sim N(0,1)$ $\textcolor{blue}{\checkmark}$
    \item[Step 4] -- \alert{Choose Critical Value $c$}
      \begin{enumerate}[(i)]
        \item Specify significance level $\alpha$.
        \item Choose $c$ so that $P(T_n > c)=\alpha$ under $H_0\colon \mu = 2$.
      \end{enumerate}
  \end{enumerate}
\end{frame}

%%%%%%%%%%%%%%%%%%%%%%%%%%%%%%%%%%%%%%%%
\begin{frame}
  \frametitle{Choose $c$ so that $P(T_n >c) = \alpha$ under $H_0$}
  $T_n = \displaystyle \left|\frac{\bar{X}-2}{0.3}\right|$ and $\mu=2 \implies \displaystyle \frac{\bar{X}-2}{0.3} \sim N(0,1)$ \pause

  \begin{eqnarray*}
    P\left( \left|\frac{\bar{X}-2}{0.3}\right| > c \right) &=& \alpha\\ \pause
    1 - P\left( \left|\frac{\bar{X}-2}{0.3}\right| \leq c \right) &=& \alpha\\ \pause
    P\left( \left|\frac{\bar{X}-2}{0.3}\right| \leq c \right) &=& 1 - \alpha\\ \pause
    P\left(-c \leq \frac{\bar{X}-2}{0.3} \leq c \right) &=& 1 - \alpha
  \end{eqnarray*}

  \alert{Hence: $c = \texttt{qnorm}(1 - \alpha/2)$ which should look familiar!}
\end{frame}
%%%%%%%%%%%%%%%%%%%%%%%%%%%%%%%%%%%%%%%%
\begin{frame}
  \frametitle{Another Simple Example: $X_1, \dots, X_{100}\sim \mbox{iid N}(\mu, \sigma^2 = 9)$}
  \begin{enumerate}
    \item[Step 1] -- $H_0\colon \mu = 2, \; H_1\colon \mu \neq 2$ $\textcolor{blue}{\checkmark}$
    \item[Step 2] -- $T_n = \displaystyle \left|\frac{\bar{X} - 2}{0.3} \right|$ $\textcolor{blue}{\checkmark}$
    \item[Step 3] -- If $\mu = 2$ then $\displaystyle \left(\frac{\bar{X} - 2}{0.3}\right) \sim N(0,1)$ $\textcolor{blue}{\checkmark}$
    \item[Step 4] -- $c = \texttt{qnorm}(1 - \alpha /2)$ $ \textcolor{blue}{\checkmark}$
    \item[Step 5] -- \alert{Look at the data: if $T_n >c$, reject $H_0$}\pause
      \begin{itemize}
        \item Suppose I choose $\alpha = 0.05$. Then $c \approx 2$.\pause
        \item I observe a sample of 100 observations. Suppose $\bar{x} = 1.34$
          \[T_n = \displaystyle\left|\frac{\bar{x} - 2}{0.3}\right| =\left|\frac{1.34 - 2}{0.3}\right| = 2.2  \]\pause
          \vspace{-2em}
        \item Since $T_n > c$, I reject $H_0\colon \mu=2$.
      \end{itemize}
  \end{enumerate}
\end{frame}

%%%%%%%%%%%%%%%%%%%%%%%%%%%%%%%%%%%%%%%%
\begin{frame}
  \frametitle{General Version of Preceding Example}

 $X_1, \dots, X_n \sim \mbox{iid N}(\mu, \sigma^2)$ with $\sigma^2$ known and we want to test:
  \[
    \begin{array}{c}
      H_0\colon \mu = \mu_0\\
      H_1\colon \mu \neq \mu_0\\
    \end{array}
  \]
  where $\mu_0$ is some specified value for the population mean.

  \pause
  
  \begin{itemize}
    \item $|\bar{X}_n - \mu_0|$ tells how far sample mean is from $\mu_0$. \pause
    \item Reject $H_0\colon \mu=\mu_0$ if sample mean is far from $\mu_0$. \pause
    \item Under $H_0\colon \mu = \mu_0$, $\displaystyle\frac{\bar{X}_n - \mu_0}{\sigma/\sqrt{n}} \sim N(0,1)$. \pause
    \item Test statistic $T_n = \displaystyle\left|\frac{\bar{X}_n - \mu_0}{\sigma/\sqrt{n}}\right|$ \pause
    \item Reject $H_0\colon \mu = \mu_0$ if $T_n > \texttt{qnorm}(1 - \alpha/2)$
  \end{itemize}

  
  
\end{frame}
%%%%%%%%%%%%%%%%%%%%%%%%%%%%%%%%%%%%%%%%
\section{Relationship Between Confidence Intervals and Hypothesis Tests}
%%%%%%%%%%%%%%%%%%%%%%%%%%%%%%%%%%%%%%%%
\begin{frame}
  \frametitle{This looks suspiciously similar to a confidence interval\dots}

  \small 
  \[
    \boxed{X_1, \dots, X_n \sim \mbox{iid N}(\mu, \sigma^2) \mbox{ where }\sigma^2 \mbox{ is known}}
  \]
  \[
    \boxed{T_n = \displaystyle\left|\frac{\bar{X}_n - \mu_0}{\sigma/\sqrt{n}}\right|, \; c = \texttt{qnorm}(1 - \alpha/2), \; \mbox{Reject } H_0\colon \mu = \mu_0 \mbox{ if } T_n > c}
  \]

  \vspace{1em}
  Another way of saying this is don't reject $H_0$ if:
  \begin{eqnarray*}
    \left(T_n \leq c\right) \pause &\iff&
    \left(\left|\frac{\bar{X}_n - \mu_0}{\sigma/\sqrt{n}}\right| \leq c \right) \pause
    \iff \left(-c \leq \frac{\bar{X}_n - \mu_0}{\sigma/\sqrt{n}}\leq c\right)\\ \pause
    &\iff& \left(\bar{X}_n - c \times \frac{\sigma}{\sqrt{n}} \leq \mu_0 \leq \bar{X}_n + c\times\frac{\sigma}{\sqrt{n}}  \right)
  \end{eqnarray*}

  \pause

  \alert{In other words, don't reject $H_0\colon \mu = \mu_0$ at significance level $\alpha$ if $\mu_0$ lies inside the $100 \times (1 - \alpha)\%$ confidence interval for $\mu$.}

\end{frame}
%%%%%%%%%%%%%%%%%%%%%%%%%%%%%%%%%%%%%%%%
\begin{frame}
  \frametitle{CIs and Hypothesis Tests are Intimately Related}

  \begin{block}{Our Simple Example}
    $X_1, \dots, X_{100} \sim \mbox{iid N}(\mu, \sigma^2 = 9)$ and observe $\bar{x} = 1.34$ 
  \end{block}

  \pause

  \begin{block}{Test $H_0\colon \mu = 2$ vs.\ $H_1\colon \mu \neq 2$ with $\alpha = 0.05$}
    $T_n = 2.2$, $c = \texttt{qnorm}(1 - 0.05/2) \approx 2$. Since $T_n>c$ we reject.
    
  \end{block}

  \pause

  \begin{block}{95\% Confidence Interval for $\mu$} 
    $1.34 \pm 2 \times 3 / 10$ i.e.\ $1.34 \pm 0.6$ or equivalently $(0.74, 1.94)$
  \end{block}

  \pause

  \begin{block}{Another way to carry out the test\dots}
   Since 2 lies outside the 95\% confidence interval for $\mu$, if our significance level is $\alpha = 0.05$ we reject $H_0\colon \mu = 2$.
  \end{block}

\end{frame}
%%%%%%%%%%%%%%%%%%%%%%%%%%%%%%%%%%%%%%%%
\begin{frame}
  \frametitle{Roadmap}

  \begin{block}{Next Time}
    More examples of hypothesis testing, using relationship with confidence intervals to help us.
  \end{block}


  \pause

  \begin{block}{Building Intuition}
   Now that you know a simple example of a hypothesis test and its relationship to a CI, think about the following:
   \begin{itemize}
     \item If we reject $H_0$ does that mean that $H_0$ is false?
     \item How does testing relate to random sampling?
      \item How does critical value of a test relate to width of a CI?
   \end{itemize}
  \end{block}

\end{frame}
%%%%%%%%%%%%%%%%%%%%%%%%%%%%%%%%%%%%%%%%
