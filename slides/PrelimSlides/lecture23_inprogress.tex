%To compile as handout, use
%pdflatex "\def\ishandout{1} \input{filename.tex}"
%Defaults to non-handout mode (with slide reveals)
\ifdefined\ishandout
  \documentclass[handout]{beamer}
\else
  \documentclass{beamer}
\fi
 
\usepackage{econ103slides} 

\date{Lecture 22}


\begin{document} 




%%%%%%%%%%%%%%%%%%%%%%%%%%%%%%%%%%%%%%%%

\begin{frame}[plain]
	\titlepage 
	

\end{frame} 

%%%%%%%%%%%%%%%%%%%%%%%%%%%%%%%%%%%%%%%%%
\begin{frame}
\frametitle{What is a Type I Error? \includegraphics[scale = 0.05]{./images/clicker}}

\begin{enumerate}[(a)]
	\item Rejecting a false Null
	\item Failing to Reject a True Null
	\item Rejecting a True Null
	\item Failing to Reject a False Null
\end{enumerate}

\end{frame}


%%%%%%%%%%%%%%%%%%%%%%%%%%%%%%%%%%%%%%%%
\begin{frame}
\frametitle{What is a Type II Error? \includegraphics[scale = 0.05]{./images/clicker}}

\begin{enumerate}[(a)]
	\item Rejecting a False Null
	\item Failing to Reject a True Null
	\item Rejecting a True Null
	\item Failing to Reject a False Null
\end{enumerate}

\end{frame}


%%%%%%%%%%%%%%%%%%%%%%%%%%%%%%%%%%%%%%%%
\begin{frame}
\frametitle{Which is the Probability of a Type I Error? \includegraphics[scale = 0.05]{./images/clicker}}

\begin{enumerate}[(a)]
	\item $\beta$
	\item $1 - \beta$
	\item $1-\alpha$
	\item $\alpha$
\end{enumerate}

\end{frame}
%%%%%%%%%%%%%%%%%%%%%%%%%%%%%%%%%%%%%%%%
\begin{frame}[c]\frametitle{The Power of a Test}
    
\begin{block}
	{Type I Error} Rejecting $H_0$ when it is true:  $P(\mbox{Type I Error}) = \alpha$
\end{block}

\begin{block}
	{Type II Error} Failing to reject $H_0$ when it is false: \alert{$P(\mbox{Type II Error}) =\beta$}
\end{block}

\begin{alertblock}
	{Statistical Power} The probability of rejecting $H_0$ when it is false: \alert{$\mbox{Power} = 1 -\beta $}\\ i.e.\ the probability of \emph{convicting} a guilty person.
\end{alertblock}

\vspace{1em}

\begin{center}
\alert{\boxed{
\begin{minipage}
	{0.95\textwidth}
	Hypothesis tests designed to control Type I error rate ($\alpha$). But we also care about Type II errors. What can learn about these?
\end{minipage}}}
\end{center}
\end{frame}
%%%%%%%%%%%%%%%%%%%%%%%%%%%%%%%%%%%%%%%%
\begin{frame}
\frametitle{Example of Calculating Power: Is this coin fair?}

\begin{block}
{Flip a Possibly Unfair Coin $n$ Times}
$X_1, \hdots, X_n \sim \mbox{iid Bernoulli}(p)$ where $p$ may not equal 1/2
\end{block}

\begin{block}
	{Test $H_0\colon p = 0.5$ against $H_1\colon p \neq 0.5$}
	Under $H_0$ and if $n$ is large the CLT gives 
	$$\displaystyle T_n = \frac{\widehat{p} - 0.5}{\sqrt{\frac{0.5(1-0.5)}{n}}}\approx N(0,1)$$
\end{block}

\begin{alertblock}
	{The Idea Behind Statistical Power}
	How does the test statistic behave if the  $H_0$ is \emph{false}? What is the sampling distribution of $T_n$ \emph{under the alternative}: $H_1\colon p\neq 0.5$
\end{alertblock}

\end{frame}
%%%%%%%%%%%%%%%%%%%%%%%%%%%%%%%%%%%%%%%%
\begin{frame}
% \frametitle{Sampling Dist.\ of $T_n$ Under the Alternative $H_1\colon p \neq 0.5$}

\begin{block}
	{Key Distributional Result}
	Center and standardize $\widehat{p}$ using \emph{true} $p \implies$ standard normal:
	$$Z = \frac{\widehat{p} - p}{\sqrt{\frac{p(1-p)}{n}}}  \approx N(0,1)$$
\end{block}

\begin{block}
	{Under the Alternative $p\neq 0.5$}
	$T_n$ is \emph{incorrectly centered and scaled}: $\displaystyle T_n = \frac{\widehat{p} - 0.5}{\sqrt{\frac{0.5(1-0.5)}{n}}} \neq N(0,1)$
\end{block}

\begin{alertblock}
	{How We'll Proceed}
	Use algebra to express $T_n$ in terms of $Z$, which we know is $N(0,1)$, and constants. This will give the distribution of $T_n$ under $H_1$.
\end{alertblock}
\end{frame}
%%%%%%%%%%%%%%%%%%%%%%%%%%%%%%%%%%%%%%%%
\begin{frame}
\frametitle{This is Just Algebra}
\small
	\begin{eqnarray*}
	T_n &=&\frac{\widehat{p} - 0.5}{\sqrt{\frac{0.5(1-0.5)}{n}}} =\sqrt{n}(2\widehat{p} - 1)\\ \\
	&=& \sqrt{n} \left[2\widehat{p} -1 + (2p -2p)\right] = \sqrt{n}\left[ 2\left(\widehat{p} - p\right) + 2p -1\right]\\ \\
	&=& \sqrt{n}\left[ 2\left(\widehat{p} - p\right) \left(\frac{\sqrt{p(1-p)/n}}{\sqrt{p(1-p)/n}}\right) + 2p -1\right]\\ \\
		&=& \sqrt{n}\left[ 2\sqrt{\frac{p(1-p)}{n}} \alert{\left(\frac{\widehat{p} - p}{\sqrt{p(1-p)/n}}\right)} + 2p -1\right]\\ \\
		&=& \left[2\sqrt{p(1-p)}\right] \alert{Z} + \sqrt{n}(2p -1)\\
		&=& a\, \alert{Z} + b
\end{eqnarray*}
\end{frame}
%%%%%%%%%%%%%%%%%%%%%%%%%%%%%%%%%%%%%%%%
\begin{frame}
\frametitle{Distribution of Test Statistic Under $H_1\colon p \neq 0.5$ }

From the previous slide,
	$$T_n =\frac{\widehat{p} - 0.5}{\sqrt{\frac{0.5(1-0.5)}{n}}} = \alert{a Z + b}$$
where:
	\begin{eqnarray*}
		Z &=& \frac{\widehat{p} - p}{\sqrt{p(1-p)/n}} \approx N(0,1)\\ 
		a &=& 2\sqrt{p(1-p)}\\ 
		b &=& \sqrt{n}(2p -1) 
	\end{eqnarray*}

\vspace{1em}
\alert{Hence: $T_n \approx N(\mu = b, \sigma^2 = a^2)$}
\end{frame}
%%%%%%%%%%%%%%%%%%%%%%%%%%%%%%%%%%%%%%%%
\begin{frame}[c]\frametitle{Distribution of $T_n$ Under the Alternative}
    
    $$\boxed{T_n =\frac{\widehat{p} - 0.5}{\sqrt{\frac{0.5(1-0.5)}{n}}} \approx N\left(\sqrt{n}(2p-1), 4p(1-p) \right)}$$

\begin{block}
	{Note That:}
	\begin{enumerate}
		\item Mean depends on $p$ and $n$
		\item Variance depends only on $p$
		\item If $p=0.5$ so $H_0$ is true, this reduces to a standard normal:
		\begin{eqnarray*}
		\mbox{Mean} &=&  \sqrt{n}(2p-1) = \sqrt{n}(2\times 1/2 - 1) = 0\\
			\mbox{Variance} &=& 4p(1-p) = 4 \times 1/2 \times (1 - 1/2) = 1\\
		\end{eqnarray*}
	\end{enumerate}
\end{block}

\end{frame}

%%%%%%%%%%%%%%%%%%%%%%%%%%%%%%%%%%%%%%%%
\begin{frame}
	\frametitle{\href{http://glimmer.rstudio.com/fditraglia/power_proptest/}{http://glimmer.rstudio.com/fditraglia/power\_proptest/}}
\framesubtitle{Ignore everything except the solid curve and play around with the first two sliders.}

\begin{figure}
	\fbox{\includegraphics[scale = 0.22]{./images/power_proptest_screenshot}}
\end{figure}

\end{frame}
%%%%%%%%%%%%%%%%%%%%%%%%%%%%%%%%%%%%%%%%
\begin{frame}
\frametitle{Now We can Calculate Power!}
\small
\begin{block}{Decision Rule for Two-sided Test}
Reject $H_0\colon p = 0.5$ provided that $\alert{|T_n| \geq \texttt{qnorm}(1 - \alpha/2)}$
\end{block}

\begin{block}{If the null is false (i.e.\ under $H_1\colon p \neq 0.5$)}
	$$T_n = \frac{\widehat{p} - 0.5}{\sqrt{\frac{0.5(1-0.5)}{n}}} \approx N\left(\sqrt{n}(2p-1), 4 p(1-p)  \right)$$
\end{block}
\begin{block}{Thus, the probability of rejecting a false null is:}
	$$\boxed{\mbox{Power}(\alpha, p, n) = P\left( |Y| \geq c\right)  = P(Y \leq -c) + P(Y\geq c)}$$
		\begin{eqnarray*}
			 c &=&\texttt{qnorm}(1 - \alpha/2) \\
			 Y &\sim& N\left(\sqrt{n}(2p-1), 4 p(1-p)  \right)
		 \end{eqnarray*} 
\end{block}
\end{frame}
%%%%%%%%%%%%%%%%%%%%%%%%%%%%%%%%%%%%%%%%
\begin{frame}
	\frametitle{\href{http://glimmer.rstudio.com/fditraglia/power_proptest/}{http://glimmer.rstudio.com/fditraglia/power\_proptest/}}
\framesubtitle{Now look at everything and try changing all the sliders!}

\begin{figure}
	\fbox{\includegraphics[scale = 0.22]{./images/power_proptest_screenshot}}
\end{figure}

\end{frame}
%%%%%%%%%%%%%%%%%%%%%%%%%%%%%%%%%%%%%%%%\begin{frame}
\begin{frame}
\begin{center}
	\includegraphics[scale=0.37]{./images/power_change_n}\\
	\includegraphics[scale=0.37]{./images/power_change_a}
\end{center}
\end{frame}
%%%%%%%%%%%%%%%%%%%%%%%%%%%%%%%%%%%%%%%%
\begin{frame}
\frametitle{Some Intuition about Power for the Coin Example}
\small
	\begin{itemize}
	\item Equals prob.\ of rejecting false null, i.e.\ convicting a guilty person.
	\item Tells us how large a sample we would need to detect a given discrepancy from ``fairness'' of the coin: $p = 0.5$
		\begin{itemize}
			\item Small deviations from $p=0.5$ unlikely to be detected unless the sample size is large.
			\item Large deviations from $p=0.5$ very likely to be detected even if the sample size is small.
		\end{itemize}
		\item For a \emph{given} degree of ``unfairness'' (deviation from $p = 0.5$)
			\begin{itemize}
				\item Higher significance level ($\alpha$) $\implies$ higher power ($1-\beta$)
				\item Large sample size ($n$) $\implies$ higher power ($1-\beta$)
			\end{itemize}
	\end{itemize}
\end{frame}

%%%%%%%%%%%%%%%%%%%%%%%%%%%%%%%%%%%%%%%%

\begin{frame}
\frametitle{Power More Generally}
	\begin{block}{Power $ = (1 - \beta) = 1 -  P$(Type II Error)}
Chance of detecting an effect given that one exists.
\end{block}
\begin{block}{Power Depends on Four Things}
	\begin{enumerate}
\item Magnitude of Effect: easier to detect large deviations from $H_0$
\item Amount of variability in the population: less variability $\implies$ easier to detect an effect of given size.
\item Sample Size: larger $n \implies$ easier to detect effect of given size
\item Signif.\ Level ($\alpha$): fewer Type I errors $\implies$ more Type II errors
\end{enumerate}
\end{block}
\alert{Go back and compare to factors that affect width of CI...}
\end{frame}
%%%%%%%%%%%%%%%%%%%%%%%%%%%%%%%%%%%%%%%%
\begin{frame}
\frametitle{An Important Point}
For any discrepancy from $p = 0.5$, there is always a sample size large enough to make the power of our test \emph{arbitrarily close to one}. In other words, we can be almost certain to detect an effect, no matter how small, provided that we use a large enough sample. \alert{However, this does not mean that the affect we have found is important}.

\vspace{2em}
If it turned out that the true probability of heads was closer to 0.50001 rather than 0.5, should we really care? 
\end{frame}
% %%%%%%%%%%%%%%%%%%%%%%%%%%%%%%%%%%%%%%%%
\begin{frame}[fragile]
\frametitle{An R Function to Calculate Power For Coin Example}
\footnotesize
	\begin{eqnarray*}
		\mbox{Power}(\alpha, p, n) &=& P\left(Y \leq -c\right) + P\left(Y\geq c\right)\\
			 c &=&\texttt{qnorm}(1 - \alpha/2) \\
			 Y &\sim& N\left(\sqrt{n}(2p-1), 4 p(1-p)  \right)
		 \end{eqnarray*} 

\begin{verbatim}
coin.power <- function(a, p, n){
    c <-  qnorm(1 - a/2)	
    mu <- sqrt(n) * (2 * p - 1) 	
    sigma <- sqrt(4 * p * (1 - p))
    
    less.than <- pnorm(-c, mean = mu, sd = sigma)
    greater.than <- 1 - pnorm(c, mean = mu, sd = sigma)
    power <- less.than + greater.than
    return(power)
}
\end{verbatim}
\end{frame}


%%%%%%%%%%%%%%%%%%%%%%%%%%%%%%%%%%%%%%%%

\begin{frame}[fragile]
\frametitle{Here's Some Code for You to Play Around With}
\footnotesize
You can use the function \texttt{coin.power} to calculate power for specific values of $p, n,$ and $\alpha$
\begin{verbatim}
coin.power(a = 0.05, p = 0.55, n = 100)
coin.power(a = 0.05, p = 0.55, n = 1000)
coin.power(a = 0.1, p = 0.55, n = 100)
coin.power(a = 0.05, p = 0.6, n = 100)
\end{verbatim}
or to make plots similar to those on slide 28
\begin{verbatim}
alternatives <- seq(from = 0, to = 1, by = 0.001)
power <- coin.power(a = 0.05, alternatives, n = 10)
plot(alternatives, power, xlab = `True p', ylab = `Power', type = `l')
\end{verbatim}
Try out different choices for $p,n$ and $\alpha$ and see what you get! I'll post this R code to save typing.

\end{frame}
%%%%%%%%%%%%%%%%%%%%%%%%%%%%%%%%%%%%%%%%



\begin{frame}
\begin{center}
	\huge Some Final Thoughts on Hypothesis Testing and Confidence Intervals
\end{center}
\end{frame}
%%%%%%%%%%%%%%%%%%%%%%%%%%%%%%%%%%%%%%%%
\begin{frame}
\frametitle{Terminology I Have Intentionally Avoided Until Now}

\begin{block}{Statistical Significance}
Suppose we carry out a hypothesis test at the $\alpha\%$ level and,  based on our data, reject the null. You will often see this situation described as ``statistical significance.''
\end{block}

\begin{block}{In Other Words...}
When people say ``statistically significant'' what they really mean is that they rejected the null hypothesis.
\end{block}
\end{frame}

%%%%%%%%%%%%%%%%%%%%%%%%%%%%%%%%%%%%%%%
\begin{frame}
\frametitle{Some Examples}

	\begin{itemize}
		\item We found a difference between the ``Hi'' and ``Lo'' groups in the anchoring experiment that was statistically significant at the 5\% level based on data from a past semester.
		\item Our 95\% CI for the proportion of US voters who know who John Roberts is did not include 0.5. Viewed as a two-sided test, we found that the difference between the true population proportion and 0.5 was statistically significant at the 5\% level.
	\end{itemize}

\end{frame}
%%%%%%%%%%%%%%%%%%%%%%%%%%%%%%%%%%%%%%%
\begin{frame}
\frametitle{Why Did I Avoid this Terminology?}
\small
\begin{block}{Statistical Significance $\neq$ Practical Importance}
	\begin{itemize}
		\item You need to understand the term ``statistically significant'' since it is widely used. A better term for the idea, however, would be ``statistically discernible''
		\item Unfortunately, many people are confuse ``significance'' in the narrow, technical sense with the everyday English word meaning ``important'' 
		\item \alert{Statistically Significant Does Not Mean Important!"}
			\begin{itemize}
				\item A difference can be practically unimportant but statistically significant.
				\item A difference can be practically important but statistically insignificant.
			\end{itemize}
	\end{itemize}
\end{block}


\end{frame}


%%%%%%%%%%%%%%%%%%%%%%%%%%%%%%%%%%%%%%%
\begin{frame}
\begin{center}
\Huge P-value Measures Strength of Evidence Against $H_0$\\ \alert{Not The Size of an Effect!}
\end{center}
\end{frame}
%%%%%%%%%%%%%%%%%%%%%%%%%%%%%%%%%%%%%%%
\begin{frame}
\frametitle{Statistically Significant but Not Practically Important}
\small
I flipped a coin 10 million times (in R) and got 4990615 heads.
\begin{block}{Test of $H_0\colon p = 0.5$ against $H_1\colon p \neq 0.5$}
$$T = \displaystyle \frac{\widehat{p} - 0.5}{\sqrt{0.5(1-0.5)/n}} \approx -5.9   \implies \alert{\mbox{ p-value } \approx 0.000000003}$$
\end{block}

\begin{block}{Approximate 95\% Confidence Interval}
 $$\widehat{p} \pm \texttt{qnorm}(1 - 0.05/2) \sqrt{\frac{\widehat{p}(1-\widehat{p})}{n}}  \implies \alert{(0.4988, 0.4994)}$$
\end{block}

\footnotesize (Such a huge sample size that refined vs.\ textbook CI makes no difference)
\large
\vspace{1em}

\alert{\fbox{Actual $p$ was 0.499}}
\end{frame}

%%%%%%%%%%%%%%%%%%%%%%%%%%%%%%%%%%%%%%%

\begin{frame}
\frametitle{Practically Important But Not Statistically Significant}
\framesubtitle{\href{http://www.amazon.com/p-value-Stories-Actually-Understand-Statistics/dp/0321629302}{\fbox{Vickers: ``What is a P-value Anyway?'' (p. 62)}}}
\footnotesize
\begin{quote}
Just before I started writing this book, a study was published reporting about a 10\% lower rate of breast cancer in women who were advised to eat less fat. If this indeed the true difference, low fat diets could reduce the incidence of breast cancer by tens of thousands of women each year -- astonishing health benefit for something as simple and inexpensive as cutting down on fatty foods. The p-value for the difference in cancer rates was 0.07 and here is the key point: this was widely misinterpreted as indicating that low fat diets don't work. For example, the \emph{New York Times} editorial page trumpeted that ``low fat diets flub a test'' and claimed that the study provided ``strong evidence that the war against all fats was mostly in vain.'' \alert{However failure to prove that a treatment is effective is not the same as proving it ineffective.}
\end{quote}
\end{frame}
%%%%%%%%%%%%%%%%%%%%%%%%%%%%%%%%%%%%%%%
\begin{frame}[c]\frametitle{Do Students with 4-Letter Surnames Do Better?}
 \framesubtitle{Based on Data from Midterm 1}
    \begin{columns}
    	\column{0.35\textwidth} \begin{block}
    		{4-Letter Surname}
    			$\bar{x} = 88.9$\\
    			$s_x = 10.4$\\
    			$n_x = 12$
    	\end{block} 
    	\column{0.35\textwidth} \begin{block}
    		{Other Surnames}
    			$\bar{y} = 74.4$\\
    			$s_y = 20.7$\\
    			$n_y = 92$
    	\end{block}
    \end{columns}

\vspace{1em}
\begin{block}
	{Difference of Means}
	$\bar{x} - \bar{y} = \alert{14.5}$
\end{block}
\begin{block}
	{Standard Error}
	$\displaystyle SE = \sqrt{s_x^2/n_x + s_y^2/n_y} \approx \alert{3.7}$
\end{block}
\begin{block}
	{Test Statistic}
	$T = 14.5 / 3.7 \approx \alert{3.9}$
\end{block}
\end{frame}
%%%%%%%%%%%%%%%%%%%%%%%%%%%%%%%%%%%%%%%
\begin{frame}[c]\frametitle{What is the p-value for the two-sided test?  \hfill \includegraphics[scale = 0.05]{./images/clicker}}
    
$$\boxed{\mbox{Test Statistic} \approx 3.9}$$

\begin{enumerate}[(a)]
	\item $p < 0.01$
	\item $0.01 \leq p < 0.05$
	\item $0.05 \leq p < 0.1$
	\item $p > 0.1$
	\item Not Sure
\end{enumerate}
\end{frame}
%%%%%%%%%%%%%%%%%%%%%%%%%%%%%%%%%%%%%%%
\begin{frame}[c]\frametitle{What do these results mean? \hfill \includegraphics[scale = 0.05]{./images/clicker}}

Evaulate this statement in light of our hypothesis test:
\vspace{1em}

\begin{quote}
	Students with four-letter long surnames do better, on average, on the first midterm of Econ 103 at UPenn.
\end{quote}

\begin{enumerate}[(a)]
	\item Strong evidence in favor
	\item Moderate evidence in favor
	\item No evidence either way
	\item Moderate evidence against
	\item Strong evidence against
\end{enumerate}
\end{frame}
%%%%%%%%%%%%%%%%%%%%%%%%%%%%%%%%%%%%%%%
\begin{frame}[c,fragile]\frametitle{I just did 134 Hypothesis Tests...}
   
 \begin{block}
 	{... and 11 of them were significant at the 5\% level.}
 \end{block}

\footnotesize

\begin{verbatim}
         group sign p.value x.bar N.x  s.x y.bar N.y  s.y
26  first1 = P    1   0.000  93.8   3  2.9  75.5 101 20.4
70     id2 = 7    1   0.000  94.6   5  3.3  75.1  99 20.4
134    id8 = 0    1   0.000  92.6   7  4.9  74.8  97 20.5
5    Nlast = 4    1   0.001  88.9  12 10.4  74.4  92 20.7
90     id4 = 8    1   0.003  87.7   9  9.0  74.9  95 20.7
105    id6 = 8    1   0.003  88.1   5  5.8  75.4  99 20.6
109    id6 = 2    1   0.007  88.9   8 10.7  75.0  96 20.6
9    Nlast = 2    1   0.016  90.4   5  9.3  75.3  99 20.5
49   last1 = P   -1   0.036  65.2   6  9.9  76.7  98 20.6
65     id2 = 1    1   0.038  84.3   9 10.1  75.3  95 20.9
117    id7 = 8    1   0.041  83.4  13 11.6  75.0  91 21.1
\end{verbatim}
\end{frame}


%%%%%%%%%%%%%%%%%%%%%%%%%%%%%%%%%%%%%%
\begin{frame}
\frametitle{Green Jelly Beans Cause Acne!}
\framesubtitle{\href{http://xkcd.com/882/}{\fbox{xkcd \#882}}}
\begin{center}
	\includegraphics[scale=0.45]{./images/xkcd1}
\end{center}
\end{frame}
%%%%%%%%%%%%%%%%%%%%%%%%%%%%%%%%%%%%%%%%
\begin{frame}
\begin{center}
	\includegraphics[scale=0.45]{./images/xkcd2}

\end{center}
\end{frame}
%%%%%%%%%%%%%%%%%%%%%%%%%%%%%%%%%%%%%%%%
\begin{frame}
\begin{center}
	\includegraphics[scale=0.45]{./images/xkcd3}

\end{center}
\end{frame}
%%%%%%%%%%%%%%%%%%%%%%%%%%%%%%%%%%%%%%%%
\begin{frame}
\begin{center}
	\includegraphics[scale=0.45]{./images/xkcd4}

\end{center}
\end{frame}
%%%%%%%%%%%%%%%%%%%%%%%%%%%%%%%%%%%%%%%%
\begin{frame}
\frametitle{\hfill \includegraphics[scale = 0.05]{./images/clicker}}
Ignoring outside information, that is strictly on the basis of the hypothesis tests presented in the cartoon, do you think we have reason to believe that green jelly beans are linked to acne?
	\begin{enumerate}[(a)]
		\item Yes
		\item No
		\item Not Sure
	\end{enumerate}

\end{frame}
%%%%%%%%%%%%%%%%%%%%%%%%%%%%%%%%%%%%%%%%
\begin{frame}
\frametitle{Data-Dredging}
\begin{itemize}
	\item Suppose you have a long list of null hypotheses and assume, for the sake of argument that all of them are true.
		\begin{itemize}
			\item E.g.\ there's no difference in grades between students with different 4th digits of their student id number. 
		\end{itemize}
	\item We'll still reject about 5\% of the null hypotheses.
	\item Academic journals tend only to publish results in which a null hypothesis is rejected at the 5\% level or lower. 
	\item We end up with the bizarre result that ``most published studies are false.''  
\end{itemize}


\alert{I posted a reading about this on Piazza: ``The Economist - Trouble in the Lab.'' To learn even more, see \href{http://www.plosmedicine.org/article/info:doi/10.1371/journal.pmed.0020124}{\textcolor{blue}{\fbox{Ioannidis (2005)}}}}


\end{frame}
%%%%%%%%%%%%%%%%%%%%%%%%%%%%%%%%%%%%%%%%
\begin{frame}
\frametitle{Green Jelly Beans Cause Acne!}
\framesubtitle{\href{http://xkcd.com/882/}{\fbox{xkcd \#882}}}
\begin{figure}
\centering
	\includegraphics[scale=0.45]{./images/xkcd1}
	\caption{Go and read this comic strip: before today's lecture you wouldn't have gotten the joke!}
\end{figure}
\end{frame}
%%%%%%%%%%%%%%%%%%%%%%%%%%%%%%%%%%%%%%%%

\begin{frame}
\frametitle{Don't Compare P-Values Across Different Tests!}
\framesubtitle{\fbox{\href{http://www.people-press.org/files/2012/08/8-10-12-Knowledge-release.pdf}{Pew: ``What Voters Know About Campaign 2012''}}}


\footnotesize

Of 239 Republicans, 61\% knew Romney is pro-life vs.\ 53\% of 238 Democrats.
\pause
\begin{block}{$H_0\colon p_{Rep} = 0.5$ vs.\ $H_1\colon p_{Rep} \neq 0.5$}
 $$T = \frac{0.61 - 0.5}{\sqrt{0.5(1-0.5)/239}} \approx  3.4 \implies \mbox{ p-value } \approx 0.0007$$
\end{block}
\pause
\begin{block}{$H_0\colon q_{Dem} = 0.5$ vs.\ $H_1\colon q_{Dem} \neq 0.5$}
 $$T = \frac{0.53 - 0.5}{\sqrt{0.5(1-0.5)/238}} \approx 0.93  \implies \mbox{ p-value } \approx 0.35$$
\end{block}
\pause
\begin{block}{$H_0\colon p_{Rep} =q_{Dem}$ vs.\ $H_1\colon p_{Rep} \neq q_{Dem}$}
 $$T = \frac{0.61 - 0.53}{\sqrt{\left(\frac{1}{239}+ \frac{1}{238}\right)\left(\frac{239 \times 0.61 + 238 \times 0.53}{239 + 238}\right)}} \approx  1.76 \implies \mbox{ p-value } \approx 0.08$$
\end{block}


\end{frame}

%%%%%%%%%%%%%%%%%%%%%%%%%%%%%%%%%%%%%%%%
\begin{frame}
\frametitle{Don't Compare P-Values Across Different Tests!}

\begin{itemize}
	\item P-Value measures strength of evidence against the null, not the size of an affect! \pause
	\item Use a single test to address a single research question: if you are actually interested in differences between Republicans and Democrats, test for this directly! \pause
\end{itemize}

\vspace{1em}

\pause

For more on the problems associated with comparing p-values from different hypothesis tests, along with an even starker example than the one I just showed you, see \href{http://amstat.tandfonline.com/doi/abs/10.1198/000313006X152649}{\textcolor{blue}{\fbox{Gelman \& Stern (2006)}}}

\end{frame}


%%%%%%%%%%%%%%%%%%%%%%%%%%%%%%%%%%%%%%%%

\begin{frame}
\frametitle{Some Final Thoughts}
	\begin{itemize}
		\item Failing to reject $H_0$ does not mean $H_0$ is true. 
		\item Rejecting $H_0$ does not mean $H_1$ is true.
		\item P-values are always more informative than simply reporting ``Reject'' vs.\ ``Fail To Reject'' at a given significance level. 
		\item Confidence intervals are more informative that hypothesis tests, since they give an idea of the size of an effect. 
		\item If $H_0$ is actually plausible a priori (this is rarer than you may think), reporting a p-value can be a good complement to a CI. 
		\item To avoid data-dredging be honest about the tests you have carried out: report \emph{all of them}, not just the ones where you rejected the null.
	\end{itemize}

\end{frame}
\end{document}