\documentclass{beamer}  
%use [handout] option to print without all the pauses!
\usepackage{setspace}
\linespread{1.3}
\usepackage{amssymb, amsmath, amsthm} 
\usepackage{rotating}
\usepackage{multirow}
\usepackage{graphicx}
\usepackage{enumerate}
\usepackage{synttree}
\usepackage{fancybox}
\usepackage{color}
\usepackage{tikz}
\usetikzlibrary{trees}
\newcommand{\p}{\mathbb{P}}
\newcommand{\expect}{\mathbb{E}}
\newcommand{\var}{\mathbb{V}}



%\setbeamertemplate{blocks}[rounded][shadow=true] 
%gets rid of bottom navigation bars
\setbeamertemplate{footline}{
   \begin{beamercolorbox}[ht=4ex,leftskip=0.3cm,rightskip=0.3cm]{author in head/foot}
%    \usebeamercolor{UniBlue}
    \vspace{0.1cm}
    %\insertshorttitle \ - \insertdate 
    \hfill \insertframenumber / \inserttotalframenumber
   \end{beamercolorbox}
   \vspace*{0.1cm}
} 

%gets rid of navigation symbols
\setbeamertemplate{navigation symbols}{}


%Include or exclude the notes?
%\setbeameroption{show notes}
\setbeameroption{hide notes}


\title[Econ 103]{Economics 103, Statistics for Economists} 
\author[F. DiTraglia]{Francis J.\ DiTraglia}
\institute{University of Pennsylvania}
\date{Lecture 23}


\begin{document} 




%%%%%%%%%%%%%%%%%%%%%%%%%%%%%%%%%%%%%%%%

\begin{frame}[plain]
	\titlepage 
	

\end{frame} 

%%%%%%%%%%%%%%%%%%%%%%%%%%%%%%%%%%%%%%%

\begin{frame}
\begin{center}
	\huge Some Final Thoughts on Hypothesis Testing and Confidence Intervals
\end{center}
\end{frame}
%%%%%%%%%%%%%%%%%%%%%%%%%%%%%%%%%%%%%%%%
\begin{frame}
\frametitle{Terminology I Have Intentionally Avoided Until Now}

\begin{block}{Statistical Significance}
Suppose we carry out a hypothesis test at the $\alpha\%$ level and,  based on our data, reject the null. You will often see this situation described as ``statistical significance.''
\end{block}

\begin{block}{In Other Words...}
When people say ``statistically significant'' what they really mean is that they rejected the null hypothesis.
\end{block}
\end{frame}

%%%%%%%%%%%%%%%%%%%%%%%%%%%%%%%%%%%%%%%
\begin{frame}
\frametitle{Some Examples}

	\begin{itemize}
		\item We found a difference between the ``Hi'' and ``Lo'' groups in the anchoring experiment that was statistically significant at the 5\% level based on data from a past semester.
		\item Our 95\% CI for the proportion of US voters who know who John Roberts is did not include 0.5. Viewed as a two-sided test, we found that the difference between the true population proportion and 0.5 was statistically significant at the 5\% level.
	\end{itemize}

\end{frame}
%%%%%%%%%%%%%%%%%%%%%%%%%%%%%%%%%%%%%%%
\begin{frame}
\frametitle{Why Did I Avoid this Terminology?}
\small
\begin{block}{Statistical Significance $\neq$ Practical Importance}
	\begin{itemize}
		\item You need to understand the term ``statistically significant'' since it is widely used. A better term for the idea, however, would be ``statistically discernible''
		\item Unfortunately, many people are confuse ``significance'' in the narrow, technical sense with the everyday English word meaning ``important'' 
		\item \alert{Statistically Significant Does Not Mean Important!"}
			\begin{itemize}
				\item A difference can be practically unimportant but statistically significant.
				\item A difference can be practically important but statistically insignificant.
			\end{itemize}
	\end{itemize}
\end{block}


\end{frame}


%%%%%%%%%%%%%%%%%%%%%%%%%%%%%%%%%%%%%%%
\begin{frame}
\begin{center}
\Huge P-value Measures Strength of Evidence Against $H_0$\\ \alert{Not The Size of an Effect!}
\end{center}
\end{frame}
%%%%%%%%%%%%%%%%%%%%%%%%%%%%%%%%%%%%%%%
\begin{frame}
\frametitle{Statistically Significant but Not Practically Important}
\small
I flipped a coin 10 million times (in R) and got 4990615 heads.
\begin{block}{Test of $H_0\colon p = 0.5$ against $H_1\colon p \neq 0.5$}
$$T = \displaystyle \frac{\widehat{p} - 0.5}{\sqrt{0.5(1-0.5)/n}} \approx -5.9   \implies \alert{\mbox{ p-value } \approx 0.000000003}$$
\end{block}

\begin{block}{Approximate 95\% Confidence Interval}
 $$\widehat{p} \pm \texttt{qnorm}(1 - 0.05/2) \sqrt{\frac{\widehat{p}(1-\widehat{p})}{n}}  \implies \alert{(0.4988, 0.4994)}$$
\end{block}

\footnotesize (Such a huge sample size that refined vs.\ textbook CI makes no difference)
\large
\vspace{1em}

\alert{\fbox{Actual $p$ was 0.499}}
\end{frame}

%%%%%%%%%%%%%%%%%%%%%%%%%%%%%%%%%%%%%%%

\begin{frame}
\frametitle{Practically Important But Not Statistically Significant}
\framesubtitle{\href{http://www.amazon.com/p-value-Stories-Actually-Understand-Statistics/dp/0321629302}{\fbox{Vickers: ``What is a P-value Anyway?'' (p. 62)}}}
\footnotesize
\begin{quote}
Just before I started writing this book, a study was published reporting about a 10\% lower rate of breast cancer in women who were advised to eat less fat. If this indeed the true difference, low fat diets could reduce the incidence of breast cancer by tens of thousands of women each year -- astonishing health benefit for something as simple and inexpensive as cutting down on fatty foods. The p-value for the difference in cancer rates was 0.07 and here is the key point: this was widely misinterpreted as indicating that low fat diets don't work. For example, the \emph{New York Times} editorial page trumpeted that ``low fat diets flub a test'' and claimed that the study provided ``strong evidence that the war against all fats was mostly in vain.'' \alert{However failure to prove that a treatment is effective is not the same as proving it ineffective.}
\end{quote}
\end{frame}
%%%%%%%%%%%%%%%%%%%%%%%%%%%%%%%%%%%%%%%
\begin{frame}[c]\frametitle{Do Students with 4-Letter Surnames Do Better?}
 \framesubtitle{Based on Data from Midterm 1}
    \begin{columns}
    	\column{0.35\textwidth} \begin{block}
    		{4-Letter Surname}
    			$\bar{x} = 88.9$\\
    			$s_x = 10.4$\\
    			$n_x = 12$
    	\end{block} 
    	\column{0.35\textwidth} \begin{block}
    		{Other Surnames}
    			$\bar{y} = 74.4$\\
    			$s_y = 20.7$\\
    			$n_y = 92$
    	\end{block}
    \end{columns}

\vspace{1em}
\begin{block}
	{Difference of Means}
	$\bar{x} - \bar{y} = \alert{14.5}$
\end{block}
\begin{block}
	{Standard Error}
	$\displaystyle SE = \sqrt{s_x^2/n_x + s_y^2/n_y} \approx \alert{3.7}$
\end{block}
\begin{block}
	{Test Statistic}
	$T = 14.5 / 3.7 \approx \alert{3.9}$
\end{block}
\end{frame}
%%%%%%%%%%%%%%%%%%%%%%%%%%%%%%%%%%%%%%%
\begin{frame}[c]\frametitle{What is the p-value for the two-sided test?  \hfill \includegraphics[scale = 0.05]{./images/clicker}}
    
$$\boxed{\mbox{Test Statistic} \approx 3.9}$$

\begin{enumerate}[(a)]
	\item $p < 0.01$
	\item $0.01 \leq p < 0.05$
	\item $0.05 \leq p < 0.1$
	\item $p > 0.1$
	\item Not Sure
\end{enumerate}
\end{frame}
%%%%%%%%%%%%%%%%%%%%%%%%%%%%%%%%%%%%%%%
\begin{frame}[c]\frametitle{What do these results mean? \hfill \includegraphics[scale = 0.05]{./images/clicker}}

Evaulate this statement in light of our hypothesis test:
\vspace{1em}

\begin{quote}
	Students with four-letter long surnames do better, on average, on the first midterm of Econ 103 at UPenn.
\end{quote}

\begin{enumerate}[(a)]
	\item Strong evidence in favor
	\item Moderate evidence in favor
	\item No evidence either way
	\item Moderate evidence against
	\item Strong evidence against
\end{enumerate}
\end{frame}
%%%%%%%%%%%%%%%%%%%%%%%%%%%%%%%%%%%%%%%
\begin{frame}[c,fragile]\frametitle{I just did 134 Hypothesis Tests...}
   
 \begin{block}
 	{... and 11 of them were significant at the 5\% level.}
 \end{block}

\footnotesize

\begin{verbatim}
         group sign p.value x.bar N.x  s.x y.bar N.y  s.y
26  first1 = P    1   0.000  93.8   3  2.9  75.5 101 20.4
70     id2 = 7    1   0.000  94.6   5  3.3  75.1  99 20.4
134    id8 = 0    1   0.000  92.6   7  4.9  74.8  97 20.5
5    Nlast = 4    1   0.001  88.9  12 10.4  74.4  92 20.7
90     id4 = 8    1   0.003  87.7   9  9.0  74.9  95 20.7
105    id6 = 8    1   0.003  88.1   5  5.8  75.4  99 20.6
109    id6 = 2    1   0.007  88.9   8 10.7  75.0  96 20.6
9    Nlast = 2    1   0.016  90.4   5  9.3  75.3  99 20.5
49   last1 = P   -1   0.036  65.2   6  9.9  76.7  98 20.6
65     id2 = 1    1   0.038  84.3   9 10.1  75.3  95 20.9
117    id7 = 8    1   0.041  83.4  13 11.6  75.0  91 21.1
\end{verbatim}
\end{frame}


%%%%%%%%%%%%%%%%%%%%%%%%%%%%%%%%%%%%%%%
% \begin{frame}
% \frametitle{Green Jelly Beans Cause Acne!}
% \framesubtitle{\href{http://xkcd.com/882/}{\fbox{xkcd \#882}}}
% \begin{center}
% 	\includegraphics[scale=0.45]{./images/xkcd1}
% \end{center}
% \end{frame}
% %%%%%%%%%%%%%%%%%%%%%%%%%%%%%%%%%%%%%%%%
% \begin{frame}
% \begin{center}
% 	\includegraphics[scale=0.45]{./images/xkcd2}

% \end{center}
% \end{frame}
% %%%%%%%%%%%%%%%%%%%%%%%%%%%%%%%%%%%%%%%%
% \begin{frame}
% \begin{center}
% 	\includegraphics[scale=0.45]{./images/xkcd3}

% \end{center}
% \end{frame}
% %%%%%%%%%%%%%%%%%%%%%%%%%%%%%%%%%%%%%%%%
% \begin{frame}
% \begin{center}
% 	\includegraphics[scale=0.45]{./images/xkcd4}

% \end{center}
% \end{frame}
% %%%%%%%%%%%%%%%%%%%%%%%%%%%%%%%%%%%%%%%%
% \begin{frame}
% \frametitle{\hfill \includegraphics[scale = 0.05]{./images/clicker}}
% Ignoring outside information, that is strictly on the basis of the hypothesis tests presented in the cartoon, do you think we have reason to believe that green jelly beans are linked to acne?
% 	\begin{enumerate}[(a)]
% 		\item Yes
% 		\item No
% 		\item Not Sure
% 	\end{enumerate}

% \end{frame}
%%%%%%%%%%%%%%%%%%%%%%%%%%%%%%%%%%%%%%%%
\begin{frame}
\frametitle{Data-Dredging}
\begin{itemize}
	\item Suppose you have a long list of null hypotheses and assume, for the sake of argument that all of them are true.
		\begin{itemize}
			\item E.g.\ there's no difference in grades between students with different 4th digits of their student id number. 
		\end{itemize}
	\item We'll still reject about 5\% of the null hypotheses.
	\item Academic journals tend only to publish results in which a null hypothesis is rejected at the 5\% level or lower. 
	\item We end up with the bizarre result that ``most published studies are false.''  
\end{itemize}


\alert{I posted a reading about this on Piazza: ``The Economist - Trouble in the Lab.'' To learn even more, see \href{http://www.plosmedicine.org/article/info:doi/10.1371/journal.pmed.0020124}{\textcolor{blue}{\fbox{Ioannidis (2005)}}}}


\end{frame}
%%%%%%%%%%%%%%%%%%%%%%%%%%%%%%%%%%%%%%%%
\begin{frame}
\frametitle{Green Jelly Beans Cause Acne!}
\framesubtitle{\href{http://xkcd.com/882/}{\fbox{xkcd \#882}}}
\begin{figure}
\centering
	\includegraphics[scale=0.45]{./images/xkcd1}
	\caption{Go and read this comic strip: before today's lecture you wouldn't have gotten the joke!}
\end{figure}
\end{frame}
%%%%%%%%%%%%%%%%%%%%%%%%%%%%%%%%%%%%%%%%

% \begin{frame}
% \frametitle{Don't Compare P-Values Across Different Tests!}
% \framesubtitle{\fbox{\href{http://www.people-press.org/files/2012/08/8-10-12-Knowledge-release.pdf}{Pew: ``What Voters Know About Campaign 2012''}}}


% \footnotesize

% Of 239 Republicans, 61\% knew Romney is pro-life vs.\ 53\% of 238 Democrats.
% \pause
% \begin{block}{$H_0\colon p_{Rep} = 0.5$ vs.\ $H_1\colon p_{Rep} \neq 0.5$}
%  $$T = \frac{0.61 - 0.5}{\sqrt{0.5(1-0.5)/239}} \approx  3.4 \implies \mbox{ p-value } \approx 0.0007$$
% \end{block}
% \pause
% \begin{block}{$H_0\colon q_{Dem} = 0.5$ vs.\ $H_1\colon q_{Dem} \neq 0.5$}
%  $$T = \frac{0.53 - 0.5}{\sqrt{0.5(1-0.5)/238}} \approx 0.93  \implies \mbox{ p-value } \approx 0.35$$
% \end{block}
% \pause
% \begin{block}{$H_0\colon p_{Rep} =q_{Dem}$ vs.\ $H_1\colon p_{Rep} \neq q_{Dem}$}
%  $$T = \frac{0.61 - 0.53}{\sqrt{\left(\frac{1}{239}+ \frac{1}{238}\right)\left(\frac{239 \times 0.61 + 238 \times 0.53}{239 + 238}\right)}} \approx  1.76 \implies \mbox{ p-value } \approx 0.08$$
% \end{block}


% \end{frame}

%%%%%%%%%%%%%%%%%%%%%%%%%%%%%%%%%%%%%%%%
% \begin{frame}
% \frametitle{Don't Compare P-Values Across Different Tests!}

% \begin{itemize}
% 	\item P-Value measures strength of evidence against the null, not the size of an affect! \pause
% 	\item Use a single test to address a single research question: if you are actually interested in differences between Republicans and Democrats, test for this directly! \pause
% \end{itemize}

% \vspace{1em}

% \pause

% For more on the problems associated with comparing p-values from different hypothesis tests, along with an even starker example than the one I just showed you, see \href{http://amstat.tandfonline.com/doi/abs/10.1198/000313006X152649}{\textcolor{blue}{\fbox{Gelman \& Stern (2006)}}}

% \end{frame}


%%%%%%%%%%%%%%%%%%%%%%%%%%%%%%%%%%%%%%%%

\begin{frame}
\frametitle{Some Final Thoughts}
	\begin{itemize}
		\item Failing to reject $H_0$ does not mean $H_0$ is true. 
		\item Rejecting $H_0$ does not mean $H_1$ is true.
		\item P-values are always more informative than simply reporting ``Reject'' vs.\ ``Fail To Reject'' at a given significance level. 
		\item Confidence intervals are more informative that hypothesis tests, since they give an idea of the size of an effect. 
		\item If $H_0$ is actually plausible a priori (this is rarer than you may think), reporting a p-value can be a good complement to a CI. 
		\item To avoid data-dredging be honest about the tests you have carried out: report \emph{all of them}, not just the ones where you rejected the null.
	\end{itemize}

\end{frame}
%%%%%%%%%%%%%%%%%%%%%%%%%%%%%%%%%%%%%%%%

\begin{frame}
\begin{center}
	\huge Regression -- Part II
\end{center}
\end{frame}
%%%%%%%%%%%%%%%%%%%%%%%%%%%%%%%%%%%%%%%%
\begin{frame}
\frametitle{Recall: ``Best Fitting'' Line Through Cloud of Points}
\begin{figure}
	\includegraphics[scale = 0.48]{./images/midterms5}
\end{figure}
\end{frame}
%%%%%%%%%%%%%%%%%%%%%%%%%%%%%%%%%%%%%%%%
\begin{frame}
\frametitle{Recall: Regression as a Data Summary}
\begin{block}{Linear Model}
$\hat{y} = a + b x$
\end{block}
\begin{block}{Choose $a,b$ to Minimize Sum of Squared Vertical Deviations}
$\displaystyle\sum_{i = 1}^n d_i^2 = \sum_{i=1}^n (y_i - a - b x_i)^2$
\end{block}

\begin{block}{The Prediction}
Predict score $\hat{y} = a + b x$ on second midterm for someone with score $x$ on first.
\end{block}

\end{frame}
%%%%%%%%%%%%%%%%%%%%%%%%%%%%%%%%%%%%%%%%
\begin{frame}
\frametitle{Recall: Regression as a Data Summary}
	\begin{block}{Problem}
	$$\min_{a,b}  \sum_{i=1}^n (y_i - a - b x_i)^2$$
\end{block}
\begin{block}{Solution}
	\begin{eqnarray*}
		b &=& \frac{\sum_{i=1}^n \left(y_i - \bar{y}\right)\left(x_i - \bar{x} \right)}{\sum_{i=1}^n \left(x_i - \bar{x}\right)^2} = \frac{s_{xy}}{s_x^2} =  r\frac{s_y}{s_x}\\ \\
		a &=& \bar{y} - b\bar{x}
	\end{eqnarray*}
\end{block}
\end{frame}
%%%%%%%%%%%%%%%%%%%%%%%%%%%%%%%%%%%%%%%%
\begin{frame}
\frametitle{Beyond Regression as a Data Summary}
\small
Based on a sample of Econ 103 students, we made the following graph of handspan against height, and fitted a linear regression:
\begin{center}
\includegraphics[scale = 0.3]{./images/height_handspan_reg}
\end{center}
The estimated slope was about 1.4 inches/cm and the estimated intercept was about 40 inches. \\


\alert{What if anything does this tell us about the relationship between height and handspan \emph{in the population}?}
\end{frame}
%%%%%%%%%%%%%%%%%%%%%%%%%%%%%%%%%%%%%%%%
\begin{frame}
	\frametitle{The Population Regression Model}
	How is $Y$ (height) related to $X$ (handspan) in the population?
	\begin{block}{Assumption I: Linearity}
		The random variable $Y$ is linearly related to $X$ according to 
			$$Y = \beta_0 + \beta_1 X + \epsilon$$
		$\beta_0,\beta_1$ are two unknown population parameters (constants).
	\end{block}

	\begin{block}
		{Assumption II: Error Term $\epsilon$}
		$E[\epsilon]=0$, $Var(\epsilon) = \sigma^2$ and $\epsilon$ is indpendent of $X$. The error term $\epsilon$ measures the unpredictability of $Y$ \emph{after controlling for $X$}
	\end{block}
\end{frame}
%%%%%%%%%%%%%%%%%%%%%%%%%%%%%%%%%%%%%%%%
\begin{frame}
	\frametitle{Predictive Interpretation of Regression}
	\begin{block}
		{Under Assumptions I and II}
		$$E[Y|X] = \beta_0 + \beta_1 X$$ 
			\begin{itemize}
				\item ``Best guess'' of $Y$ having observed $X=x$ is $\beta_0 + \beta_1 x$
				\item If $X = 0$, we predict $Y = \beta_0$
				\item If two people differ by one unit in $X$, we predict that they will differ by $\beta_1$ units in $Y$.
			\end{itemize}
	\end{block}
	\alert{The only problem is, we don't know $\beta_0, \beta_1$...}
\end{frame}
%%%%%%%%%%%%%%%%%%%%%%%%%%%%%%%%%%%%%%%%
\begin{frame}
	\frametitle{Estimating $\beta_0, \beta_1$}
Suppose we observe an iid sample $(Y_1, X_1), \hdots, (Y_n, X_n)$ from the population: $Y_i = \beta_0 + \beta_1 X_i + \epsilon_i$. Then we can \emph{estimate} $\beta_0, \beta_1$:

	$$\widehat{\beta}_1 = \frac{\sum_{i=1}^n (X_i - \bar{X}_n) (Y_i - \bar{Y}_n)}{\sum_{i=1}^n (X_i - \bar{X}_n)^2}$$

	
	$$\widehat{\beta}_0 = \bar{Y}_n - \widehat{\beta}_1 \bar{X}_n$$

\vspace{2em}
	\alert{Once we have estimators, we can think about sampling uncertainty...}
\end{frame}

%%%%%%%%%%%%%%%%%%%%%%%%%%%%%%%%%%%%%%%%

\begin{frame}
\frametitle{Sampling Uncertainty: Pretend the Class is our Population}

\begin{figure}[h]
\centering
\includegraphics[scale = 0.5]{./images/height_handspan_reg}
\caption{Estimated Slope = 1.4, Estimated Intercept = 40}
\end{figure}


\end{frame}
%%%%%%%%%%%%%%%%%%%%%%%%%%%%%%%%%%%%%%%%
\begin{frame}
\frametitle{Sampling Distribution of Regression Coefficients $\widehat{\beta}_0$ and $\widehat{\beta}_1$}
\vspace{1em}

\begin{center}
\setlength{\unitlength}{1cm}
\begin{picture}(5,7)
\put(-2,6){\framebox(9,1){Choose 25 Students from Class List with Replacement}}



\put(0.5,6){\vector(-1,-1){1.5}}
\put(-2.3,3.7){\framebox(2.5,0.65){Sample 1}}



\put(-1,3.5){\vector(0,-1){1}}
\put(-1.6,1.9){\makebox{\small $\widehat{\beta}_0^{(1)}, \widehat{\beta}_1^{(1)}$}}



\put(2,6){\vector(0,-1){1.5}}
\put(0.7,3.7){\framebox(2.5,0.65){Sample 2}}



\put(2,3.5){\vector(0,-1){1}}
\put(1.4,1.9){\makebox{\small $\widehat{\beta}_0^{(2)}, \widehat{\beta}_1^{(2)}$}}



\put(3.8,4){\makebox{...}}
\put(3.8,2){\makebox{...}}



\put(4.5,6){\vector(1,-1){1.5}}
\put(4.8,3.7){\framebox(2.5,0.65){Sample 1000}}



\put(6,3.5){\vector(0,-1){1}}
\put(5.3,1.9){\makebox{\small $\widehat{\beta}_0^{(1000)}, \widehat{\beta}_1^{(1000)}$}}



\put(-2,0.6){\makebox{\small Repeat 1000 times $\rightarrow$  get 1000 different pairs of estimates}}



\put(-2,0.1){\makebox{\small \alert{Sampling Distribution: long-run relative frequencies}}}

\end{picture}
\end{center}


\end{frame}


%%%%%%%%%%%%%%%%%%%%%%%%%%%%%%%%%%%%%%%%

\begin{frame}
\frametitle{1000 Replications, $n=25$}

\begin{figure}[h]
\centering
\includegraphics[scale = 0.6]{./images/reg_lines_sim}

\end{figure}


\end{frame}

%%%%%%%%%%%%%%%%%%%%%%%%%%%%%%%%%%%%%%%%

\begin{frame}
\frametitle{Population: Intercept = 40, Slope = 1.4}

\begin{figure}[h]
\centering
\includegraphics[scale = 0.55]{./images/reg_sim_hist}

\end{figure}
\alert{Based on 1000 Replications, $n = 25$}

\end{frame}

%%%%%%%%%%%%%%%%%%%%%%%%%%%%%%%%%%%%%%%%
\begin{frame}
\frametitle{Inference for Linear Regression}
	\begin{block}{Central Limit Theorem}
		$$\frac{\widehat{\beta} - \beta}{\widehat{SE}(\widehat{\beta})} \approx N(0,1)$$ 
\end{block}

\begin{block}{How to calculate $\widehat{SE}$?}
	\begin{itemize}
\item Complicated 
	\begin{itemize}
\item Depends on variance of errors $\epsilon$ and all predictors in regression. 
\item We'll look at a few simple examples 
\item R does this calculation for us 
\end{itemize}
\item Requires assumptions about population errors $\epsilon_i$ 
	\begin{itemize}
\item Simplest (and R default) is to assume $\epsilon_i \sim iid (0,\sigma^2)$ 
\item Weaker assumptions in Econ 104 
\end{itemize}

\end{itemize}
\end{block}

\end{frame}
%%%%%%%%%%%%%%%%%%%%%%%%%%%%%%%%%%%%%%%%
\begin{frame}
\frametitle{Intuition for What Effects $SE(\widehat{\beta}_1)$ for Simple Regression}

	$$SE(\widehat{\beta}_1) \approx \frac{\sigma}{\sqrt{n}} \cdot \frac{1}{s_X}$$
	\begin{itemize}
	\item $\sigma = SD(\epsilon)$ -- inherent variability of the $Y$, even after controlling for $X$
	\item $n$ is the sample size
	\item $s_X$ is the sampling variability of the $X$ observations.
	\end{itemize}
\end{frame}


%%%%%%%%%%%%%%%%%%%%%%%%%%%%%%%%%%%%%%%%
\begin{frame}
I treated the class as our population for the purposes of the simulation experiment but it makes more sense to think of the class as a sample from some population. We'll take this perspective now and think about various inferences we can draw from the height and handspan data using regression.
\end{frame}




%%%%%%%%%%%%%%%%%%%%%%%%%%%%%%%%%%%%%%%%
\begin{frame}[fragile]
\frametitle{Height = $\beta_0 +  \epsilon$}
\footnotesize
\begin{verbatim}
lm(formula = height ~ 1, data = student.data)
            coef.est coef.se
(Intercept) 67.74     0.51  
---
n = 80, k = 1
\end{verbatim}
\pause
\begin{verbatim}
> mean(student.data$height)
[1] 67.7375
\end{verbatim}
\pause
\begin{verbatim}
> sd(student.data$height)/sqrt(length(student.data$height))
[1] 0.5080814
\end{verbatim}
\end{frame}

%%%%%%%%%%%%%%%%%%%%%%%%%%%%%%%%%%%%%%%%
\begin{frame}
\frametitle{Dummy Variable (aka Binary Variable)}
 
A predictor variable that takes on only two values: 0 or 1. Used to represent two categories, e.g.\ Male/Female.
\end{frame}


%%%%%%%%%%%%%%%%%%%%%%%%%%%%%%%%%%%%%%%%

\begin{frame}[fragile]
\frametitle{Height = $\beta_0 + \beta_1$ Male $+ \epsilon$}

\footnotesize
\begin{verbatim}
lm(formula = height ~ sex, data = student.data)
            coef.est coef.se
(Intercept) 64.46     0.56  
sexMale      6.10     0.76  
---
n = 80, k = 2
residual sd = 3.38, R-Squared = 0.45
\end{verbatim}
\pause
\begin{verbatim}
> mean(male$height) - mean(female$height)
[1] 6.09868
\end{verbatim}
\pause
\begin{verbatim}
> sqrt(var(male$height)/length(male$height) +
		 			var(female$height)/length(female$height))
[1] 0.7463796
\end{verbatim}
\end{frame}


%%%%%%%%%%%%%%%%%%%%%%%%%%%%%%%%%%%%%%%%


\begin{frame}[fragile]
\frametitle{Height = $\beta_0 + \beta_1$ Male $+ \epsilon$ \hfill \includegraphics[scale = 0.05]{./images/clicker}}

\alert{What is the ME for an approximate 95\% confidence interval for the difference of population means of height: (men - women)?}

\footnotesize
\begin{verbatim}
lm(formula = height ~ sex, data = student.data)
            coef.est coef.se
(Intercept) 64.46     0.56  
sexMale      6.10     0.76  
---
n = 80, k = 2
residual sd = 3.38, R-Squared = 0.45
\end{verbatim}

\end{frame}


%%%%%%%%%%%%%%%%%%%%%%%%%%%%%%%%%%%%%%%%



\begin{frame}[fragile]
\frametitle{Height = $\beta_0 + \beta_1$ Handspan $+ \epsilon$}
\footnotesize
\begin{verbatim}
lm(formula = height ~ handspan, data = student.data)
            coef.est coef.se
(Intercept) 39.60     3.96  
handspan     1.36     0.19  
---
n = 80, k = 2
residual sd = 3.56, R-Squared = 0.40
\end{verbatim}

\end{frame}
%%%%%%%%%%%%%%%%%%%%%%%%%%%%%%%%%%%%%%%%
\begin{frame}[fragile]
\frametitle{Height = $\beta_0 + \beta_1$ Handspan $+ \epsilon$ \hfill \includegraphics[scale = 0.05]{./images/clicker}}
\alert{What is the ME for an approximate 95\% CI for $\beta_1$?}
\footnotesize
\begin{verbatim}
lm(formula = height ~ handspan, data = student.data)
            coef.est coef.se
(Intercept) 39.60     3.96  
handspan     1.36     0.19  
---
n = 80, k = 2
residual sd = 3.56, R-Squared = 0.40
\end{verbatim}
\end{frame}



\begin{frame}
\begin{center}
\Huge Bring Your Laptop Next Time: We'll be Using R
\end{center}

\end{frame}


%%%%%%%%%%%%%%%%%%%%%%%%%%%%%%%%%%%%%%%%
\end{document}





































