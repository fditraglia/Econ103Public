%%%%%%%%%%%%%%%%%%%%%%%%%%%%%%%%%%%%%%%%
\section{The Standard Normal RV}
%%%%%%%%%%%%%%%%%%%%%%%%%%%%%%%%%%%%%%%%
\begin{frame}
  \frametitle{Available on Etsy, Made using R!}
\begin{figure}
\includegraphics[scale = 0.2]{./images/normal_etsy1}
\caption{Standard Normal RV (PDF)}
\end{figure}
\end{frame}

%%%%%%%%%%%%%%%%%%%%%%%%%%%%%%%%%%%%%%%%
\begin{frame}

\frametitle{Standard Normal RV: PDF at left, CDF at right}



\begin{figure}[h]
  \centering
  \begin{tabular}{cc}
  \includegraphics[scale = 0.3]{./images/std_normal_PDF}
  &  
  \includegraphics[scale = 0.3]{./images/std_normal_CDF}
\end{tabular}
\end{figure}


\begin{itemize}
  \item Notation: $X \sim N(0,1)$
  \item Support Set $= (-\infty,\infty)$
  \item PDF symmetric about 0, bell-shaped
  \item $E[X]=0$, $Var[X]=1$
  \item For Econ 103, don't need formula for PDF.
  \item No closed-form expression for CDF.
\end{itemize}
\end{frame}

%%%%%%%%%%%%%%%%%%%%%%%%%%%%%%%%%%%%%%%%

\begin{frame}
	\frametitle{\href{https://fditraglia.shinyapps.io/normal_cdf/}{https://fditraglia.shinyapps.io/normal\_cdf/}}

\begin{figure}
	\fbox{\includegraphics[scale = 0.2]{./images/normal_cdf_screenshot}}
\end{figure}

\end{frame}

%%%%%%%%%%%%%%%%%%%%%%%%%%%%%%%%%%%%%%%%
\begin{frame}
  \frametitle{R Commands for the Standard Normal RV}

\begin{table}
\centering
\fbox{\begin{tabular}{ll}
  PDF $f(x)$& \texttt{dnorm(x)}\\
CDF $F(x)$&\texttt{pnorm(x)}\\
  Make \texttt{n} Random Draws & \texttt{rnorm(n)}
\end{tabular}}
\end{table}

\begin{block}{Mnemonic}
  \begin{itemize}
    \item \texttt{norm} = ``Normal'' 
    \item \texttt{d} = ``density''
    \item \texttt{p} = ``probability'' 
    \item \texttt{r} = ``random''
  \end{itemize}
\end{block}

\end{frame}
%%%%%%%%%%%%%%%%%%%%%%%%%%%%%%%%%%%%%%%%
\begin{frame}[fragile]
\begin{knitrout}
\definecolor{shadecolor}{rgb}{0.969, 0.969, 0.969}\color{fgcolor}\begin{kframe}
\begin{alltt}
\hlkwd{par}\hlstd{(}\hlkwc{mfrow} \hlstd{=} \hlkwd{c}\hlstd{(}\hlnum{1}\hlstd{,} \hlnum{2}\hlstd{))}
\hlkwd{curve}\hlstd{(}\hlkwd{dnorm}\hlstd{(x),} \hlopt{-}\hlnum{4}\hlstd{,} \hlnum{4}\hlstd{,} \hlkwc{main} \hlstd{=} \hlstr{'N(0,1) PDF'}\hlstd{)}
\hlkwd{curve}\hlstd{(}\hlkwd{pnorm}\hlstd{(x),} \hlopt{-}\hlnum{4}\hlstd{,} \hlnum{4}\hlstd{,} \hlkwc{main} \hlstd{=} \hlstr{'N(0,1) CDF'}\hlstd{)}
\end{alltt}
\end{kframe}
\includegraphics[width=\maxwidth]{figure/normal_plots-1} 
\begin{kframe}\begin{alltt}
\hlkwd{par}\hlstd{(}\hlkwc{mfrow} \hlstd{=} \hlkwd{c}\hlstd{(}\hlnum{1}\hlstd{,} \hlnum{1}\hlstd{))}
\end{alltt}
\end{kframe}
\end{knitrout}
\end{frame}
%%%%%%%%%%%%%%%%%%%%%%%%%%%%%%%%%%%%%%%%
\begin{frame}[fragile]
\begin{knitrout}
\definecolor{shadecolor}{rgb}{0.969, 0.969, 0.969}\color{fgcolor}\begin{kframe}
\begin{alltt}
\hlkwd{set.seed}\hlstd{(}\hlnum{1234}\hlstd{)}
\hlstd{normal_sims} \hlkwb{<-} \hlkwd{rnorm}\hlstd{(}\hlnum{10000}\hlstd{)}

\hlkwd{mean}\hlstd{(normal_sims)}
\end{alltt}
\begin{verbatim}
## [1] 0.006115893
\end{verbatim}
\begin{alltt}
\hlkwd{var}\hlstd{(normal_sims)}
\end{alltt}
\begin{verbatim}
## [1] 0.9752143
\end{verbatim}
\end{kframe}
\end{knitrout}
\end{frame}
%%%%%%%%%%%%%%%%%%%%%%%%%%%%%%%%%%%%%%%%
\begin{frame}[fragile]
\begin{knitrout}
\definecolor{shadecolor}{rgb}{0.969, 0.969, 0.969}\color{fgcolor}\begin{kframe}
\begin{alltt}
\hlkwd{hist}\hlstd{(normal_sims,} \hlkwc{freq} \hlstd{=} \hlnum{FALSE}\hlstd{)}
\end{alltt}
\end{kframe}
\includegraphics[width=\maxwidth]{figure/normal_hist-1} 

\end{knitrout}
\end{frame}
%%%%%%%%%%%%%%%%%%%%%%%%%%%%%%%%%%%%%%%%
%\begin{frame}
%  \frametitle{$\Phi(x_0)$ Denotes the $N(0,1)$ CDF}
%  You will sometimes encounter the notation $\Phi(x_0)$.
%  It means the same thing as $\texttt{pnorm}(x_0)$ but it's not an R command. 
%\end{frame}
%%%%%%%%%%%%%%%%%%%%%%%%%%%%%%%%%%%%%%%%
\section{Linear Combinations and the $N(\mu, \sigma^2)$ RV}
%%%%%%%%%%%%%%%%%%%%%%%%%%%%%%%%%%%%%%%%
\begin{frame}
  \frametitle{$Y \sim N(\mu, \sigma^2)$ Random Variable}

  \begin{block}{Linear Function of $N(0,1)$}
    Let $X \sim N(0,1)$ and define $Y = \mu + \sigma X$ where $\mu,\sigma$ are constants. 
  \end{block}


  \begin{block}{Properties of $N(\mu, \sigma^2)$}
    \begin{itemize}
      \item Parameters: $\mu$, $\sigma^2$.
      \item Support Set $=(-\infty,\infty)$
      \item PDF symmetric about $\mu$, bell-shaped. 
      \item Special case: $N(0,1)$ has $\mu=0$ and $\sigma^2 = 1$. 
    \end{itemize}
  \end{block}

  \alert{What are the mean and variance of a $N(\mu, \sigma^2)$? How do we know?}
\end{frame}
%%%%%%%%%%%%%%%%%%%%%%%%%%%%%%%%%%%%%%%%
\begin{frame}
  \frametitle{Expected Value: $\mu$ \emph{shifts} PDF}
\framesubtitle{all of these have $\sigma=1$}

\begin{figure}
\includegraphics[scale = 0.5]{./images/normal_means}
\caption{\textcolor{blue}{Blue $\mu = -1$},
Black $\mu = 0$,
\textcolor{red}{Red $\mu = 1$}}
\end{figure}
\end{frame}

%%%%%%%%%%%%%%%%%%%%%%%%%%%%%%%%%%%%%%%%


\begin{frame}
  \frametitle{Standard Deviation: $\sigma$ \emph{scales} PDF}
\framesubtitle{all of these have $\mu=0$}

\begin{figure}
\includegraphics[scale = 0.5]{./images/normal_std_devs}
\caption{\textcolor{blue}{Blue $\sigma^2 = 4$},
Black $\sigma^2=1$,
\textcolor{red}{Red $\sigma^2= 1/4$}}
\end{figure}
\end{frame}

%%%%%%%%%%%%%%%%%%%%%%%%%%%%%%%%%%%%%%%%


\begin{frame}
\frametitle{Linear Function of Normal RV is a Normal RV}
\framesubtitle{Let $a,b$ be constants with $b \neq 0$}

\[
  \boxed{X \sim N(\mu, \sigma^2) \implies (a + bX) \sim N(a + b\mu, b^2 \sigma^2)}
\]

\begin{alertblock}{Key Point}
  Linear transformation of a normal RV is \emph{also} a normal RV!
\end{alertblock}

\end{frame}

%%%%%%%%%%%%%%%%%%%%%%%%%%%%%%%%%%%%%%%%
\begin{frame}
\frametitle{Example \hfill \includegraphics[scale = 0.05]{./images/clicker}}
Suppose $X \sim N(\mu, \sigma^2)$ and let $Z = (X -\mu)/\sigma$. What is the distribution of $Z$?

\begin{enumerate}[(a)]
	\item $N(\mu, \sigma^2)$
	\item $N(\mu, \sigma)$
	\item $N(0, \sigma^2)$
	\item  $N(0, \sigma)$
	\item $N(0,1)$
\end{enumerate}
\end{frame}

%%%%%%%%%%%%%%%%%%%%%%%%%%%%%%%%%%%%%%%%
\begin{frame}
  \frametitle{Linear Combinations of \emph{Multiple Independent} Normals}
  \framesubtitle{Let $a,b,c$ be constants and at least one of $a,b$ nonzero.} 

$X \sim N(\mu_x, \sigma^2_x)$ is independent of $Y \sim N(\mu_y, \sigma^2_y)$ then

$$\boxed{aX + bY +c \sim N(a\mu_x + b\mu_y + c, a^2 \sigma_x^2 + b^2 \sigma_y^2)}$$



\begin{alertblock}{Key Points}
	\begin{itemize}
		\item Result assumes independence
		\item Extends to more than two Normal RVs
	\end{itemize}
\end{alertblock}

\end{frame}
%%%%%%%%%%%%%%%%%%%%%%%%%%%%%%%%%%%%%%%%
\begin{frame}
\frametitle{Suppose $X_1, X_2, \sim \mbox{iid } N(\mu, \sigma^2)$ \hfill \includegraphics[scale = 0.05]{./images/clicker}}

Let $\bar{X} = (X_1 + X_2)/2$. What is the distribution of $\bar{X}$?
\begin{enumerate}[(a)]
\item $N(\mu, \sigma^2/2)$
\item $N(0,1)$
\item $N(\mu, \sigma^2)$
\item $N(\mu, 2\sigma^2)$
\item $N(2\mu, 2\sigma^2)$
\end{enumerate}

\end{frame}
%%%%%%%%%%%%%%%%%%%%%%%%%%%%%%%%%%%%%%%%
\begin{frame}
\frametitle{The ``Empirical Rule'' Gives Probabilities for a Normal RV!}

\begin{block}{Empirical Rule}
Approximately 68\% of observations within $\mu\pm \sigma$\\
Approximately 95\% of observations within $\mu\pm 2 \sigma$\\
Nearly all observations within $\mu\pm 3 \sigma$
\end{block}

\vspace{1em}

\begin{alertblock}{If $X \sim N(\mu, \sigma^2)$, then:}
  \vspace{-3em}
  \begin{align*}
    P(\mu - \sigma \leq X \leq \mu + \sigma) &\approx 0.683\\
  P(\mu - 2\sigma \leq X \leq \mu + 2\sigma) &\approx 0.954\\
  P(\mu - 3\sigma \leq X \leq \mu + 3\sigma) &\approx 0.997
  \end{align*}
\end{alertblock}

\end{frame}

%%%%%%%%%%%%%%%%%%%%%%%%%%%%%%%%%%%%%%%%
\begin{frame}[fragile]

  \fbox{For a continuous RV, $\displaystyle P(a \leq X \leq b) = \int_{a}^b f(x)\; dx = F(b) - F(a)$}

\begin{knitrout}
\definecolor{shadecolor}{rgb}{0.969, 0.969, 0.969}\color{fgcolor}\begin{kframe}
\begin{alltt}
\hlkwd{pnorm}\hlstd{(}\hlnum{1}\hlstd{)} \hlopt{-} \hlkwd{pnorm}\hlstd{(}\hlopt{-}\hlnum{1}\hlstd{)}  \hlcom{# Approx. 68% Prob. in (-1,1)}
\end{alltt}
\begin{verbatim}
## [1] 0.6826895
\end{verbatim}
\begin{alltt}
\hlkwd{pnorm}\hlstd{(}\hlnum{2}\hlstd{)} \hlopt{-} \hlkwd{pnorm}\hlstd{(}\hlopt{-}\hlnum{2}\hlstd{)}  \hlcom{# Approx. 95% Prob. in (-2,2)}
\end{alltt}
\begin{verbatim}
## [1] 0.9544997
\end{verbatim}
\begin{alltt}
\hlkwd{pnorm}\hlstd{(}\hlnum{3}\hlstd{)} \hlopt{-} \hlkwd{pnorm}\hlstd{(}\hlopt{-}\hlnum{3}\hlstd{)}  \hlcom{# > 99% Prob. in (-3,3)}
\end{alltt}
\begin{verbatim}
## [1] 0.9973002
\end{verbatim}
\end{kframe}
\end{knitrout}

\end{frame}
%%%%%%%%%%%%%%%%%%%%%%%%%%%%%%%%%%%%%%%%
\begin{frame}[t]

  \vspace{-3em}
\begin{figure}
\includegraphics[scale = 0.65]{./images/middle68_1}
\end{figure}

\vspace{-1em}
\centering 
\texttt{pnorm(1)}$\approx 0.84$
\end{frame}
%%%%%%%%%%%%%%%%%%%%%%%%%%%%%%%%%%%%%%%%
\begin{frame}[t,noframenumbering]
  \vspace{-3em}
\begin{figure}
\includegraphics[scale = 0.65]{./images/middle68_2}
\end{figure}
\vspace{-1em}
\centering
\texttt{pnorm(1) - pnorm(-1)}$\approx 0.84 - 0.16$
\end{frame}
%%%%%%%%%%%%%%%%%%%%%%%%%%%%%%%%%%%%%%%%
\begin{frame}[t,noframenumbering]
  \vspace{-3em}
\begin{figure}
\includegraphics[scale = 0.65]{./images/middle68_3}
\end{figure}
\vspace{-1em}
\centering
\texttt{pnorm(1) - pnorm(-1)}$\approx 0.68$
\end{frame}
%%%%%%%%%%%%%%%%%%%%%%%%%%%%%%%%%%%%%%%%
\begin{frame}[t,noframenumbering]
  \vspace{-3em}
\begin{figure}
\includegraphics[scale = 0.65]{./images/normal_middle68}
\end{figure}
\vspace{-1em}
\centering
Middle 68\% of $N(0,1) \Rightarrow$ approx.\ $(-1,1)$
\end{frame}

%%%%%%%%%%%%%%%%%%%%%%%%%%%%%%%%%%%%%%%%
\section{Transforming to a Standard Normal}
%%%%%%%%%%%%%%%%%%%%%%%%%%%%%%%%%%%%%%%%
\begin{frame}[t]
  \frametitle{Transforming to a Standard Normal: Example \#1}
  \fbox{Suppose $X \sim N(\mu = 1, \sigma^2 = 4)$. What is $P(-1 \leq X \leq 3)$?}

  \vspace{1em}

  \begin{alertblock}{Key Point}
    If $X \sim N(\mu, \sigma^2)$ then $\frac{X - \mu}{\sigma} \sim N(0,1)$.
  \end{alertblock}

  \begin{eqnarray*}
    P(-1 \leq X \leq 3) &=& \pause P(-2 \leq X - 1 \leq 2) \\
    &=& \pause P\left( -1 \leq \frac{X - 1}{2} \leq 1 \right)\\
    &=& \pause \texttt{pnorm(1) - pnorm(-1)}\\
    &\approx& 0.68
  \end{eqnarray*}

\end{frame}
%%%%%%%%%%%%%%%%%%%%%%%%%%%%%%%%%%%%%%%%%
\begin{frame}[t]
  \frametitle{Transforming to a Standard Normal: Example \#2}

  \fbox{Suppose $X \sim N(3,16)$. What is $P(X \geq 10)$?}

  \vspace{1em}

  \begin{alertblock}{Key Point}
    If $X \sim N(\mu, \sigma^2)$ then $\frac{X - \mu}{\sigma} \sim N(0,1)$.
  \end{alertblock}

  \begin{eqnarray*}
    P(X \geq 10) &=& \pause 1 - P(X \leq 10)\\
    &=& \pause 1 - P(X - 3 \leq 7) \\
    &=& \pause 1 - P\left( \frac{X - 3}{4} \leq \frac{7}{4} \right)\\
    &=& \texttt{1 - pnorm(7/4)} \approx 0.04
  \end{eqnarray*}

\end{frame}
%%%%%%%%%%%%%%%%%%%%%%%%%%%%%%%%%%%%%%%%%
%\begin{frame}
%\frametitle{What if $X \sim N(\mu, \sigma^2)$?}
%\begin{eqnarray*}
%	P(X \leq a) &=& \pause P(X - \mu \leq a - \mu)\\\\
%		&=& \pause P\left( \frac{X-\mu}{\sigma} \leq \frac{a - \mu}{\sigma} \right)\\\\
%		&=&\pause  P\left(Z \leq  \frac{a - \mu}{\sigma}\right)\\
%    &=& \texttt{pnorm}\left( \frac{a - \mu}{\sigma} \right)
%\end{eqnarray*}
%Where $Z$ is a standard normal random variable, i.e.\ $N(0,1)$.
%\end{frame}
%
%
%%%%%%%%%%%%%%%%%%%%%%%%%%%%%%%%%%%%%%%%%
%\begin{frame}
%\frametitle{\includegraphics[scale = 0.05]{./images/clicker}}
%Which of these equals $P\left(Z \leq (a-\mu)/\sigma\right)$ if $Z\sim N(0,1)$?
%	\begin{enumerate}[(a)]
%    \item $\texttt{pnorm(a)}$
%    \item $1 - \texttt{pnorm(a)}$
%    \item $\texttt{pnorm(a)}/\sigma - \mu$
%    \item $\texttt{pnorm}\left(\frac{a - \mu}{\sigma}  \right)$
%		\item None of the above.
%	\end{enumerate}
%\end{frame}

%%%%%%%%%%%%%%%%%%%%%%%%%%%%%%%%%%%%%%%%
%\begin{frame}
%  \frametitle{Probability \emph{Above} a Threshold: $X \sim N(\mu, \sigma^2)$}
%\begin{eqnarray*}
%	P(X \geq b) &=&1 - P(X\leq b) =1 - P\left( \frac{X-\mu}{\sigma} \leq \frac{b-\mu}{\sigma} \right) \\ \\
%	&=& 1 - P\left( Z \leq \frac{b-\mu}{\sigma} \right) \\
%	&=& 1 -\mbox{\texttt{pnorm}}((b-\mu)/\sigma)
%\end{eqnarray*}
%Where $Z$ is a standard normal random variable.
%\end{frame}
%%%%%%%%%%%%%%%%%%%%%%%%%%%%%%%%%%%%%%%%%
%\begin{frame}
%  \frametitle{Suppose $X \sim N(\mu = 1, \sigma^2 = 4)$}
%What is $P(-1 \leq X \leq 3)$?
%
%\pause
%\begin{eqnarray*}
%P(-1 \leq X \leq 4) &=& P\left( -2 \leq X - 1 \leq 2\right)\\\\
%&=& P\left( -1 \leq \frac{X-1}{2} \leq 1\right)\\ \\
%	&=& P\left( -1 \leq Z \leq 1\right)\\
%	&=& \mbox{\texttt{pnorm(1)}} -  \mbox{\texttt{pnorm(-1)}}\\
%	&\approx& 0.68
%\end{eqnarray*}
%\end{frame}
%
%%%%%%%%%%%%%%%%%%%%%%%%%%%%%%%%%%%%%%%%%
%\begin{frame}
%\frametitle{Probability of an Interval: $X \sim N(\mu, \sigma^2)$}
%
%
%\begin{eqnarray*}
%	P(a \leq X \leq b) &=&  P\left( \frac{a - \mu}{\sigma} \leq \frac{X - \mu}{\sigma} \leq \frac{b-\mu}{\sigma} \right)\\ \\ 
%	&=& P\left( \frac{a - \mu}{\sigma} \leq Z \leq \frac{b-\mu}{\sigma} \right)\\ \\ 
%	&=&\ \mbox{\texttt{pnorm}}((b-\mu)/\sigma) -  \mbox{\texttt{pnorm}}((a-\mu)/\sigma)
%\end{eqnarray*}
%Where $Z$ is a standard normal random variable.
%\end{frame}
%%%%%%%%%%%%%%%%%%%%%%%%%%%%%%%%%%%%%%%%%
\section{Percentiles/Quantiles for Continuous RVs}
%%%%%%%%%%%%%%%%%%%%%%%%%%%%%%%%%%%%%%%%

\begin{frame}
\frametitle{Quantile Function of a Continuous RV}
\framesubtitle{Quantiles are also known as Percentiles}

\begin{block}{CDF $F(x_0)$}
  \begin{itemize}
    \item $\displaystyle F(x_0) \equiv P(X \leq x_0) = \int_{-\infty}^{x_0} f(x)\, dx$
    \item Input threshold $x_0$, get probability that $X \leq x_0$.
  \end{itemize}
\end{block}

\begin{alertblock}{Quantile Function $Q(p)$}
  \begin{itemize}
\item $Q(p) = F^{-1}(p)$
\item Input probability $p$, get threshold $x_0$ such that $P(X\leq x_0) = p$.
\item In other words: $\displaystyle p = \int_{-\infty}^{x_0} f(x)\, dx$ 
  \end{itemize}
\end{alertblock}

	
\end{frame}
%%%%%%%%%%%%%%%%%%%%%%%%%%%%%%%%%%%%%%%%
\begin{frame}[t]
\frametitle{The Median of a Continuous RV}

\begin{columns}


\column{0.45\textwidth}
\vspace{-3em}
\begin{block}{Median $= Q(0.5)$}
  Median is the threshold $x_0$ such that $P(X \leq x_0) = 0.5$. 
\end{block}

\begin{block}{Median of $N(\mu, \sigma^2)$ RV}
  Normal RV is symmetric about $\mu$ so its median is $\mu$.
\end{block}

\column{0.55\textwidth}
\vspace{-3em}
\begin{figure}
\centering
\includegraphics[scale = 0.5]{./images/normal_median}
\caption{Median of $N(0,1)$ is zero.}
\end{figure}

\end{columns}

\end{frame}
%%%%%%%%%%%%%%%%%%%%%%%%%%%%%%%%%%%%%%%%
\begin{frame}
  \frametitle{R Commands for the Standard Normal RV}

\begin{table}
\centering
\fbox{\begin{tabular}{ll}
  PDF $f(x)$& \texttt{dnorm(x)}\\
CDF $F(x)$&\texttt{pnorm(x)}\\
\alert{Quantile Function $Q(p)$} & \alert{\texttt{qnorm(p)}}\\
  Make \texttt{n} Random Draws & \texttt{rnorm(n)}
\end{tabular}}
\end{table}

\begin{block}{Mnemonic}
  \begin{itemize}
    \item \texttt{norm} = ``Normal'' 
    \item \texttt{d} = ``density''
    \item \texttt{p} = ``probability'' 
    \item \texttt{r} = ``random.''
    \item \alert{\texttt{q} = ``quantile''}
  \end{itemize}
\end{block}

\end{frame}
%%%%%%%%%%%%%%%%%%%%%%%%%%%%%%%%%%%%%%%%
\begin{frame}[fragile]
\vspace{-1em}
\begin{center}
\includegraphics[scale = 0.45]{./images/normal90}
\end{center}
\vspace{-2em}
\begin{knitrout}
\definecolor{shadecolor}{rgb}{0.969, 0.969, 0.969}\color{fgcolor}\begin{kframe}
\begin{alltt}
\hlkwd{qnorm}\hlstd{(}\hlnum{0.9}\hlstd{)}  \hlcom{# 90th Percentile of Standard Normal}
\end{alltt}
\begin{verbatim}
## [1] 1.281552
\end{verbatim}
\begin{alltt}
\hlkwd{pnorm}\hlstd{(}\hlnum{1.281552}\hlstd{)}  \hlcom{# Check our answer using the CDF}
\end{alltt}
\begin{verbatim}
## [1] 0.9000001
\end{verbatim}
\end{kframe}
\end{knitrout}

\end{frame}
%%%%%%%%%%%%%%%%%%%%%%%%%%%%%%%%%%%%%%%%
\section{Symmetric Intervals for the $N(0,1)$ RV}
%%%%%%%%%%%%%%%%%%%%%%%%%%%%%%%%%%%%%%%%
\begin{frame}[t]
\frametitle{If $X\sim N(0,1)$, for what $c$ is $P(-c\leq X \leq c) = 0.5$?}

\vspace{-1.5em}
\begin{center}
\includegraphics[scale = 0.55]{./images/tail1}
\end{center}

\vspace{-2em}
\centering

\end{frame}
%%%%%%%%%%%%%%%%%%%%%%%%%%%%%%%%%%%%%%%%
\begin{frame}[t,noframenumbering]
\frametitle{If $X\sim N(0,1)$, for what $c$ is $P(-c\leq X \leq c) = 0.5$?}

\vspace{-1.5em}
\begin{center}
\includegraphics[scale = 0.55]{./images/tail4}
\end{center}

\vspace{-2em}
\centering
\textcolor{blue}{50\% Probability in Blue}; \textcolor{red}{50\% Probability in Red} \\
Boundaries of blue region are $(-c,c)$

\end{frame}
%%%%%%%%%%%%%%%%%%%%%%%%%%%%%%%%%%%%%%%%
\begin{frame}[t,noframenumbering]
\frametitle{If $X\sim N(0,1)$, for what $c$ is $P(-c\leq X \leq c) = 0.5$?}

\vspace{-1.5em}
\begin{center}
\includegraphics[scale = 0.55]{./images/tail4}
\end{center}

\vspace{-2em}
\centering
Symmetric Interval: \textcolor{red}{each red region has 25\% probability}\\
Boundaries of blue region are $(-c,c)$

\end{frame}
%%%%%%%%%%%%%%%%%%%%%%%%%%%%%%%%%%%%%%%%
\begin{frame}[t,noframenumbering]
\frametitle{If $X\sim N(0,1)$, for what $c$ is $P(-c\leq X \leq c) = 0.5$?}

\vspace{-1.5em}
\begin{center}
\includegraphics[scale = 0.55]{./images/tail3}
\end{center}

\vspace{-2em}
\centering
Let's find the right-hand boundary: $c$ 
\end{frame}
%%%%%%%%%%%%%%%%%%%%%%%%%%%%%%%%%%%%%%%%
\begin{frame}[t,noframenumbering]
\frametitle{If $X\sim N(0,1)$, for what $c$ is $P(-c\leq X \leq c) = 0.5$?}

\vspace{-1.5em}
\begin{center}
\includegraphics[scale = 0.55]{./images/tail3}
\end{center}

\vspace{-2em}
\centering
\textcolor{red}{25\% Probability to the right of $c$}\\
\textcolor{blue}{Hence, 75\% to the left of $c$}
\end{frame}
%%%%%%%%%%%%%%%%%%%%%%%%%%%%%%%%%%%%%%%%
\begin{frame}[t,noframenumbering]
\frametitle{If $X\sim N(0,1)$, for what $c$ is $P(-c\leq X \leq c) = 0.5$?}

\vspace{-1.5em}
\begin{center}
\includegraphics[scale = 0.55]{./images/tail2}
\end{center}

\vspace{-2em}
\centering
\textcolor{blue}{For what $c$ is 75\% of the probability to the left of $c$?}\\
\end{frame}
%%%%%%%%%%%%%%%%%%%%%%%%%%%%%%%%%%%%%%%%
\begin{frame}[t,noframenumbering]
\frametitle{If $X\sim N(0,1)$, for what $c$ is $P(-c\leq X \leq c) = 0.5$?}

\vspace{-1.5em}
\begin{center}
\includegraphics[scale = 0.55]{./images/tail2}
\end{center}

\vspace{-2em}
\centering

\fbox{\texttt{qnorm(0.75)} $\approx$ \texttt{0.67}}\\
\pause
\textcolor{blue}{Therefore $c = 0.67$!}

\end{frame}
%%%%%%%%%%%%%%%%%%%%%%%%%%%%%%%%%%%%%%%%
\begin{frame}[t,noframenumbering]
\frametitle{If $X\sim N(0,1)$, for what $c$ is $P(-c\leq X \leq c) = 0.5$?}

\vspace{-1.5em}
\begin{center}
\includegraphics[scale = 0.55]{./images/tail5}
\end{center}

\vspace{-2em}
\centering

Checking our work: \pause \fbox{\texttt{pnorm(0.67) - pnorm(-0.67)} $\approx$ \texttt{0.5}} \alert{\checkmark}

\end{frame}
%%%%%%%%%%%%%%%%%%%%%%%%%%%%%%%%%%%%%%%%
%
%\begin{frame}
%\frametitle{\texttt{qnorm(0.75)}$\approx 0.67$}
%Suppose $X$ is a standard normal RV. What is the value of $c$ such that $P(-c \leq X\leq c ) = 0.5$?
%\begin{center}
%\includegraphics[scale = 0.55]{./images/tail2}
%\end{center}
%\end{frame}
%
%%%%%%%%%%%%%%%%%%%%%%%%%%%%%%%%%%%%%%%%%
%\begin{frame}
%\frametitle{\texttt{qnorm(0.75)}$\approx 0.67$}
%Suppose $X$ is a standard normal RV. What is the value of $c$ such that $P(-c \leq X\leq c ) = 0.5$?
%\begin{center}
%\includegraphics[scale = 0.55]{./images/tail3}
%\end{center}
%\end{frame}
%
%%%%%%%%%%%%%%%%%%%%%%%%%%%%%%%%%%%%%%%%%
%
%\begin{frame}
%\frametitle{\texttt{pnorm(0.67)-pnorm(-0.67)}$\approx$?}
%Suppose $X$ is a standard normal RV. What is the value of $c$ such that $P(-c \leq X\leq c ) = 0.5$?
%\begin{center}
%\includegraphics[scale = 0.55]{./images/tail4}
%\end{center}
%\end{frame}
%
%%%%%%%%%%%%%%%%%%%%%%%%%%%%%%%%%%%%%%%%%
%
%
%\begin{frame}
%\frametitle{\texttt{pnorm(0.67)-pnorm(-0.67)}$\approx 0.5$}
%Suppose $X$ is a standard normal RV. What is the value of $c$ such that $P(-c \leq X\leq c ) = 0.5$?
%\begin{center}
%\includegraphics[scale = 0.55]{./images/tail5}
%\end{center}
%\end{frame}
%%%%%%%%%%%%%%%%%%%%%%%%%%%%%%%%%%%%%%%%%
%%\begin{frame}
%%\frametitle{95\% Central Interval for Standard Normal \hfill \includegraphics[scale = 0.05]{./images/clicker}}
%%
%%Suppose $X$ is a standard normal random variable. What value of $c$ ensures that $P(-c \leq X \leq c) \approx \alert{0.95}$?
%%
%%\end{frame}
%%
%
%%%%%%%%%%%%%%%%%%%%%%%%%%%%%%%%%%%%%%%%%%
%%
%%\begin{frame}
%%  \frametitle{R Commands for \emph{Arbitrary} Normal RVs: $X \sim N(\mu, \sigma^2)$}
%%\begin{table}
%%\centering
%%\fbox{\begin{tabular}{ll}
%%  PDF $f(x)$& \texttt{dnorm(x, mean = $\mu$, sd = $\sigma$)}\\
%%  Random Draws & \texttt{rnorm(n, mean = $\mu$, sd = $\sigma$)}\\
%%CDF $F(x)$&\texttt{pnorm(x, mean = $\mu$,  sd = $\sigma$)}\\
%%Quantile Function $Q(p)$ & \texttt{qnorm(p, mean = $\mu$,  sd = $\sigma$)}\\
%%\end{tabular}}
%%\end{table}
%%\vspace{1em}
%%\alert{Notice that this means you don't have to transform $X$ to a standard normal in order to find areas under its pdf using R.}
%%\end{frame}
%%%%%%%%%%%%%%%%%%%%%%%%%%%%%%%%%%%%%%%%%%
%%\begin{frame}
%%\frametitle{Example: $X \sim N(0,16)$}
%%
%%One Way:
%%			\begin{eqnarray*}
%%				P(X \geq 10) &=&  1 - P(X \leq 10) = 1 - P(X /4\leq 10/4)\\
%%				&=& 1 - P(Z\leq 2.5) =  1 - \mbox{\texttt{pnorm(2.5)}}\\ 
%%				&\approx& 0.006
%%			\end{eqnarray*}
%%\pause
%%An Easier Way:
%%	\begin{eqnarray*}
%%	P(X \geq 10) &=& 1 - P(X \leq 10)\\ 
%%	&=&  1 - \texttt{pnorm(10, mean = 0, sd = 4)} \\ 
%%	&\approx& 0.006
%%	\end{eqnarray*}
%%\end{frame}
%%%%%%%%%%%%%%%%%%%%%%%%%%%%%%%%%%%%%%%%%%
