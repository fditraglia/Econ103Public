\documentclass[11pt, letterpaper]{article}
\usepackage{geometry}
\geometry{margin=1in} 
\usepackage{setspace}
\linespread{1}
\usepackage{hyperref}
\usepackage{enumerate}
\usepackage{fancybox}
\usepackage{amsmath, amssymb}




%%% HEADERS & FOOTERS
\usepackage{fancyhdr} 
\pagestyle{fancy} % options: empty , plain , fancy
\renewcommand{\headrulewidth}{0.4pt} % customise the layout...
\rhead{\footnotesize DiTraglia -- Spring 2016}
\lhead{\footnotesize Econ 103 Syllabus}
\renewcommand\footrulewidth{0pt}


\begin{document}


\thispagestyle{plain}

\begin{center}
\Large
\sc
Statistics for Economists\\
\large
Economics 103\\
\large
Fall 2016
\end{center}


\normalsize
\bigskip
\noindent \textbf{Course Instructor:} Francis DiTraglia 

\medskip


\noindent \textbf{Recitation Instructors:}
  Matt Cook,  
  Rossa O'Keeffe-O'Donovan,
  Rodrigo Azuero-Melo
\medskip


\noindent \textbf{Office Hours:} \emph{For times and locations, see the semester calendar on the course website.}


\medskip
 
\noindent \textbf{Course Website:} \url{http://ditraglia.com/Econ103Public} At this url you can view the semester calendar and download all lecture slides, problem sets, etc.
You can view your grades and log-on to the course discussion board, Piazza, at \url{https://canvas.upenn.edu}.

\medskip

\noindent \textbf{Course Email:} Please direct all email related to this course to \href{mailto:econ103penn@gmail.com}{\texttt{econ103penn@gmail.com}} rather than your instructor or RI's personal account.
Email is reserved for \emph{personal issues only}. 
Please direct all questions concerning course material or logistics to the course discussion board, Piazza.

\medskip



\noindent \textbf{Course Description:} 
This course will teach you how to learn from data and understand uncertainty using the ideas of probability theory and statistics. 
After completing this course you will be able to carry out simple statistical analyses of your own using the computer package R.


\medskip

\noindent \textbf{Lecture Recordings: } 
Audio and screen captures of all lectures will be automatically recorded and posted on \href{http://upenn.instructure.com}{Canvas}. 
This is a great way to get caught up if you miss a lecture.
\medskip

\noindent \textbf{Prerequisites:} 
The prerequisites for this course are Math 104 followed by 114 or 115. 
To help you determine if this course is right for you we will administer a short math quiz early in the semester.
Although it will not count towards your grade, you must pass this quiz to be allowed to take the first midterm.
You may re-take the math quiz as many times as necessary until you pass.



\medskip

\noindent \textbf{Required Text:} 
The textbook for this course is \emph{Introductory Statistics for Business and Economics}, 4\textsuperscript{th} Edition by Thomas H.\ and Ronald J.\ Wonnacott (WW4). 
If you purchased your copy from the Penn bookstore, you got ripped off! Return it and get a cheap used copy on \href{http://tinyurl.com/ECON103-2013A}{Amazon}.
The document ``Recommended Readings'' on the course website provides reading assignments to accompany each lecture.
While I strongly suggest that you complete the assigned readings, my lecture slides, which will be posted online after each lecture, are the final authority on course material. 
You are \emph{not} responsible for material in the textbook \emph{unless} it is also covered in lecture, but you \emph{are} responsible for material from lecture even if it is \emph{not} covered in the textbook.

\medskip


%\noindent \textbf{Required Technology: } 
%We will be using the ``clickers'' for experiments and class participation exercises during the semester.
%Both the ResponseCard NXT and the ResponseCard QT are fully compatible with the exercises we will complete this semester.
%Other ResponseCard models will only work for some of the exercises so to make sure you get full participation credit (see below) make sure you have the right model.
%You can buy or rent a clicker from the Penn Bookstore.
%As clicker participation will make up 5\% of your course grade it is important that you bring your clicker to each lecture. 
%I fully understand, however, that things can go wrong: your clicker might stop working or you might forget to bring it after pulling an all-nighter.
%For this reason you will \emph{automatically} be excused from clicker participation for four lectures: there is no need to inform us in advance or after the fact. There will, however, be no further exceptions. 
%I will begin keeping track of clicker participation in our second lecture. For more details, see ``Participation'' below. 
%Because clickers will determine a portion of your grade, their use is subject to the code of academic integrity, as explained below under ``Academic Integrity.'' 
%


\noindent \textbf{Required Software:} 
We will use the statistical package R via a front-end called RStudio throughout the course. 
Both R and RStudio are free and open source. Installation instructions appear on the last page of this syllabus.
RStudio is also available in the Undergraduate Data Analysis Lab (UDAL) in McNeil rooms 104 and 108--9. 
You will be taught to use R primarily through a series of tutorials that I will assign as homework. (See ``Homework'' below.)  
Additional R resources are listed on the last page of this syllabus.

\medskip

\noindent \textbf{Recommended Texts:} 
I recommend two texts for students who need extra help with the course material. 
First is the \emph{Student Workbook to accompany Introductory Statistics for Business and Economics 4\textsuperscript{th} Edition}. 
Used copies are available on \href{http://www.amazon.com/gp/offer-listing/0471508993/sr=/qid=/ref=olp_page_2?ie=UTF8&colid=&coliid=&condition=all&me=&qid=&shipPromoFilter=0&sort=sip&sr=&startIndex=10}{Amazon}. 
The workbook contains fully worked out solutions to all odd-numbered problems from the textbook along with additional practice problems and solutions.
If you're having trouble with R and prefer a printed book to the free online resources listed below, I suggest consulting \emph{The R Student Companion} by Brian Dennis.


\newpage

\noindent \textbf{Departmental Course Policies: } 
All Economics Department course policies are in force in Econ 103 even if not explicitly listed on this syllabus. 
See: \url{http://economics.sas.upenn.edu/undergraduate-program/course-information/guidelines/policies} for full details. 


\bigskip


\noindent \textbf{Academic Integrity: } 
All suspected violations of the code of academic integrity as set forth in the Pennbook will be reported to the Office of Student Conduct. 
Confirmed violations will result in failure for the course. 
Because it will be used to determine your class participation grade, operating a clicker on behalf of another student is cheating. 
If you are discovered using a clicker other than your own or have votes in a class that you did not attend, you will face the penalties described above. 
We will check identification cards at exams so please be sure to bring yours.

\medskip

\noindent \textbf{Piazza:} 
We will be using an online discussion forum called Piazza for this course, which you can access directly from \href{http://upenn.instructure.com}{Canvas}. 
Piazza is where we will make course announcements and answer questions about course material and logistics.
By asking your question and getting an answer on Piazza, you create a positive externality: other students benefit from your questions and you benefit from theirs. 
The instructor and RIs will actively moderate Piazza both to answer questions and approve (or correct) answers written by your fellow-students.
If you opt-in to the participation component of the course you will receive credit for contributing answers, questions, and notes to Piazza. 
See below under ``Assignments and Grading'' for details.

\medskip

\noindent \textbf{Homework:} 
I have posted a number of problem sets and R Tutorials (with full solutions) on the course website.
Throughout the semester I will indicate which problem sets and tutorials you should complete in a given week.
Although homework will neither be collected nor graded, it is \emph{crucial} that you keep up with the homework on a weekly basis if you hope to do well in the course.
Be sure to use the solution keys responsibly: you gain nothing by simply reading them.



\section*{Assignments and Grading}

Grades for this course will be determined based on quizzes, two midterms, a final examination and, optionally, a participation component.
You have a choice between two grading schemes: the ``Default'' weights, which do not include participation, or the ``Participation Weights'' which do. 

\bigskip 

\noindent \textbf{Default Weights:} If you choose the default weights your overall score for the course will be calculated according to
	\begin{equation*}
	\begin{split}
		\mbox{Overall Score} = (20\% \times \mbox{Quizzes})  + (20\% \times \mbox{Midterm 1}) + (20\% \times \mbox{Midterm 2}) + (40\% \times \mbox{Final})
	\end{split}
	\end{equation*}
As the name suggests, the Default Weights are what we will use to calculate your grade unless you opt-in to the participation weights described in the next section.

\bigskip

\noindent \textbf{Participation Weights:} If you choose the participation weights, your overall score for the course will be calculated according to
	\begin{equation*}
	\begin{split}
		\mbox{Overall Score} = (&10\% \times \mbox{Participation}) + (20\% \times \mbox{Quizzes})  \\ &
							 + (20\% \times \mbox{Midterm 1}) + (20\% \times \mbox{Midterm 2}) + (30\% \times \mbox{Final})
	\end{split}
	\end{equation*}
Compared to the default, this option reallocates 10\% of the weight from your final to class and discussion board participation.
The participation weights are opt-in only.
The procedure for opting in, along with information about how participation scores are calculated, appear below under ``Participation.''
If you choose to opt in, this decision is final and cannot be revoked.
If you are unsure which weights to choose, I suggest the participation weights. 

\newpage

\noindent \textbf{Course Curve:}
If necessary, I will curve overall course scores (\emph{not} individual assignments) so that approximately 20-30\% fall in the A-range, 40-50\% fall in the B-range, and the bulk of the remaining 20-40\% fall in the C-range. 
I reserve grades below a C-minus for those cases in which a student fails to attain a minimum level of basic competence in statistics, an absolute rather than relative standard. 
If you are in danger of failing to meet this minimum standard, you will receive a course problem notice.
I will only curve the course in your favor, so the most stringent possible grade boundaries are: A-range = 90-100, B-range = 80-89, C-range = 70-79, D-range = 60-69.
(In this case, the top two points of each range would be a ``plus`` and the bottom two points a ``minus``.)

\medskip


\noindent \textbf{Quizzes:} 
Your RIs will administer a number of short quizzes in recitation over the course of the semester: dates appear on the semester calendar on the course website.
Each quiz will cover basic material from the most recent lectures since the last quiz or midterm. 
When calculating your quiz average, I will drop your two lowest scores and weight the remaining quizzes evenly. 
No makeup quizzes will be given so use your two ``free skips'' carefully.
Quizzes will not be returned and answers will not be posted but the RI's will go over each quiz in recitation.

\medskip

\noindent \textbf{Exams:} 
There will be two 70-minute in-class midterm exams and a 2-hour final exam during the exam period.
Dates, times and locations will appear on the semester calendar on the course website.
Each midterm is worth 20\% and the final is worth 30\% of your grade.
Neither midterm is comprehensive, but the final is: it will focuse on the final third of the course but include several questions on earlier material.
To give you a sense of the style and level of difficulty to expect, I have posted all of my past exams with full solutions on the course website.
Attendance at all exams is \emph{mandatory} and there will be no makeup midterms.
In exceptional circumstances, e.g.\ a death in the family or a serious documented illness, please contact the instructor in advance via the course email address.
The makeup final will take place at the beginning of next semester and is outside of the instructor's control: eligibility as well as the time and date are determined by the Economics Department. 
Exam regrade requests must be made in writing within a week of receiving your graded exam. 
As we re-grade the entire exam, your score could rise or fall. 
You may not discuss your answers with an RI or the instructor before submitting a regrade request. 
Exams will be photocopied before being returned and you may write in pencil or pen. 
Scientific calculators are permitted but graphing calculators are not. 
We will check ID cards at each exam.

\medskip

\noindent \textbf{Participation:} If you choose the participation weights, described above, a portion of your overall score will be determined from class and discussion board (Piazza) participation. 
To opt-in to the participation weights, you must register your mobile phone for use with Poll Everywhere by the end of the Course Selection Period.
(See ``Class Participation'' below.)
To be clear: if you register your phone, then you have opted in and you cannot revoke this decision later in the semester.
I will provide registration details both in-class and on Piazza.
Each of the two components, class and discussion board participation, will receive equal weight in the calculation.
Participation is designed to reward regular, good-faith effort over the course of the semester.
As such, it will fairly easy to obtain 100\% for this portion of your course grade.
Details regarding each of the components of participation along with the procedure for opting in to the participation track appear in the following three sections.

\medskip 

\noindent \textbf{Class Participation:}
To obtain class participation credit you will need attend class, arrive on time, pay attention, and participate in in-class activities using your mobile phone using Poll Everywhere.
As long as you have a mobile phone that supports SMS messaging you'll be able to use Poll Everywhere. 
I will purchase a license for the whole class so there is nothing that you need to buy.
You will, however, need to register your phone online so that I can link your name with your poll responses.
I will provide details for how to do this both in class and on Piazza.
If you register, I will assume that you wish to opt-in to the participation weights. 
You may revoke this decision later in the semester.
I will begin counting class participation after the end of the course selection period so you must register by then.
You do not need to have perfect attendance to get full credit for class participation.
As a rough guide, if you attend more than 80\% of the course lectures and participate actively you can expect to get 100\% for this component of your participation grade.
A fraction of the in-class activities will consist of simple multiple choice review or ``pop quiz'' questions to make sure that you are following along with the lecture material.
For these questions, I will record whether you got the answer right or wrong to check whether you are paying attention.
Fear not: this is only to foil those students who in past semesters sat in the back of the lecture hall, ignored me and entered random responses while reading Facebook.
If you're in class and paying attention, you have nothing to worry about.
Please do not come to me with absence reports for particular lectures: missing a couple of lectures here or there will have no effect on your participation score.


            
\section*{Installing R and RStudio} First, download and install R from \url{http://cran.r-project.org/}. Second, download and install RStudio by visiting \url{http://rstudio.org/download/desktop} and clicking the link listed under ``Recommended for Your System.'' 


\section*{Additional R Resources} 
\begin{itemize}
		       \item Contributed Documentation -- Comprehensive R Archive Network (CRAN) \\\url{http://cran.r-project.org/other-docs.html}
           	\begin{quote}
           		Comprehensive list of freely available reference material for R.
           	\end{quote}
\item R Twotorials -- Anthony Damico \\\url{http://www.twotorials.com/}
		\begin{quote}
		Ninety energetic, two-minute video tutorials on statistical programming with R. 
		\end{quote}
			\item Google Developers R Programming Video Lectures\\ \url{http://www.r-bloggers.com/google-developers-r-programming-video-lectures/}\begin{quote}R Programming video tutorials from beginning to advanced. \end{quote}
		 	\item Econometrics in R -- Grant Farnsworth\\\url{http://cran.r-project.org/doc/contrib/Farnsworth-EconometricsInR.pdf}
 		\begin{quote}
 		If you'd like to keep using R in Econ 104, this is what you should read.
 		\end{quote}
 			\item Resources to help you learn R -- UCLA Academic Technology Services \\\url{http://www.ats.ucla.edu/stat/R/}
		\begin{quote}
			A wealth of information about R, conveniently arranged in one place.
		\end{quote}
	           \item R in a Nutshell -- Adler\\ \url{http://proquestcombo.safaribooksonline.com/book/programming/r/9781449377502}         
           	\begin{quote}
           		E-book accessible on the UPenn Network. A comprehensive reference guide to R.
           	\end{quote}
%		\item R-bloggers \\\url{http://www.r-bloggers.com}
%		\begin{quote}
%			A blog aggregator for R news and tutorials, with lots of applications.
%		\end{quote}
\end{itemize}

\end{document}
