%        File: ClassicalProbability.tex
%     Created: Sat Jan 31 04:00 PM 2015 E
% Last Change: Sat Jan 31 04:00 PM 2015 E
%
\documentclass[12pt]{article}
\usepackage{amsmath,amssymb}
\title{Basic Combinatorics and Classical Probability\\ \large Addendum to Lecture \# 5}
\author{Econ 103}
\begin{document}
\maketitle

\section*{Introduction}
In lecture I don't spend much time on Classical Probability since I expect that this material should be familiar from High School.
If it's not familiar, or if you'd like a refresher, this document should help.
Since Classical Probability is all about counting, we'll start off with a short review of combinatorics.

\section*{Basic Combinatorics: Rules for Counting}
\subsection*{The Multiplication Rule}
When I visit Chipotle, I'm presented with a sequence of independent choices:
\begin{enumerate}
  \item Burrito, bowl, soft tacos, or crispy tacos? (4 options)
  \item White or brown rice? (2 options)
  \item Black or pinto beans? (2 options)
  \item Chicken, carnitas, steak, barbacoa, sofritas, or vegetables? (6 options) 
  \item Mild, medium or hot salsa? (3 options)
  \item Cheese? (2 options: yes/no)
  \item Guacamole? (2 options: yes/no)
\end{enumerate}
I've left off a number of the toppings, but you get the idea.
The question is: how many different meals could I order?
To answer this, we'll use a principle called the \emph{multiplication rule}.

Imagine a gigantic rectangular classroom, College Hall Room 200 perhaps, with 50 rows of seats, each containing 20 chairs.
Now suppose that I asked you to count the number of chairs in the room.
Would you visit each row individually and count each chair? 
Of course not: you'd simply multiply 50 by 20 to get 200.
This is just a special case of the multiplication rule.

Let me phrase the question a little differently.
How many different ways could you choose where to sit if this lecture hall were completely empty?
We already know that the answer is 200, but let's try getting it a different way.
The room is rectangular, so each chair is uniquely identified by its \emph{column} and \emph{row}.
Choosing a seat is equivalent to choosing a row and a column.
There are 50 ways to choose a row, and \emph{for each of these} there are 20 ways to choose a column.
This means that, if we want the total number of ways to choose, we need to repeatedly add the number 20 a total of 50 times.
But that's just the definition of multiplication, so the answer is 200.

Now suppose I added one more choice: there are \emph{three} identical classrooms, each with 50 rows of 20 seats.
If all the seats are empty, how many ways are there to decide where you want to sit?
Well, we already know how to solve the problem for \emph{each} classroom so to find the total, we just need to add up this answer three times.
But again, that is simply the definition of multiplication, so our answer is $3 \times 20 \times 50 = 600$.

Are you starting to see the pattern here?
This has nothing in particular to do with chairs in classrooms, but it's a basic feature of \emph{how we count things}: \textbf{if you have a series of independent decisions to make, the total number of ways to decide equals the product of the number of ways to make each individual choice}.
Stated more mathematically, if you have $k$ independent decisions to make and there are $n_1$ ways to make the first decision, $n_2$ ways to make the second decision, $\hdots$, and $n_k$ ways to make the $k^{th}$ decision, then there are $n_1 \times n_2 \times \cdots \times n_k$ total ways to decide.
The \emph{key requirement here} is that the decisions are \emph{independent}.
What I mean by this is that the number of ways to make a particular decision does not depend on what you chose in a \emph{separate decision}.
Going back to the classroom example, this is the same as requiring that each room have the same number of rows and columns of chairs.
Since the Chipotle example also has this structure, we can solve it as follows: $4 \times 2 \times 2 \times 6 \times 3 \times 2 \times 2 = 1152$.
That's a lot of possibilities!

If you find the multiplication rule confusiong, try thinking about some other examples of how we count things in real life.
It's very important to have a good intuitive grasp of this idea since all the other rules of counting are really just special cases of the multiplication rule.

\subsection*{Example: Lining up for a Photo}
Suppose that you're taking a picture of your three best friends: Alice, Bob, Charlotte.
Feeling unimaginative, you decide to have them stand side-by-side in a line.
How many different ways are there to set up the picture?

Although it's very simple, this example raises an important point about counting that comes up over and over again in probability questions: before we can decide \emph{how many} of something we have, we need to be clear about what gets to ``count'' as a distinct object. 
In posing the question about taking a photograph, I am implicitly considering \emph{different orderings} of your four friends as different ways to set up the picture.
The key distinction here is whether or not order matters.
In this case it clearly does: we consider (Alice, Bob, Charlotte) to be a different ``item'' than (Charlotte, Bob, Alice).
In mathematics, this is like the difference between a \emph{set} and an \emph{ordered pair}: $(1,2)$ and $(2,1)$ are different points in the Cartesian coordinate system but $\left\{ 1,2 \right\}$ and $\left\{ 2,1 \right\}$ represent the same set since order doesn't matter in set theory.

Now back to our problem.
How can we use the multiplication rule here?
The trick is to break it into a series of decisions.
Essentially, we are trying to allocate three objects to three slots:
\begin{equation*}
  \underset{1}{\underbar{\quad}}\;
  \underset{2}{\underbar{\quad}}\;
  \underset{3}{\underbar{\quad}}\;
\end{equation*}
so our decisions are as follows:
\begin{enumerate}
  \item Who goes in slot \#1? (3 options)
  \item Who goes in slot \#2? (2 options)
  \item Who goes in slot \#3? (1 options)
\end{enumerate}
Using the multiplication rule, we see that there are $3 \times 2 \times 1 = 6$ possible ways to arrange your friends in the picture.
The key is noticing that, each time we make a decision in this example, we have one fewer person who could go in the next slot.
(This makes it a little different from the Chipotle example, although we can still apply the multiplication rule.)

If you find this confusing, try directly enumerating all of the possibilities.
This example is simple enough that it won't take too long.
First suppose that Alice is in position \#1.
Then there are two possibilities: either Bob is in position \#2 and Charlotte is in position \#3 or the reverse.
Now suppose that Bob is in position \#1.
Then there are two possibilities: either Alice is in position \#2 and Charlotte is in position \#3 or the reverse.
Finally, suppose that Charlotte is in position \#1.
Then there are two possibilities: either Alice is in position \#2 and Bob is in position \#3 or the reverse.
Counting up each of these possibilites, we get six as expected.

It's no accident that the answer to this question can be expressed as $3!$.
Using exactly the same reasoning as I gave above, we can reach the following general conclusion: \textbf{there are $k!$ ways to arrange $k$ objects in order}.

\subsection*{Permutations}
Permutations are almost exactly like the example from the preceding section in which we arranged three people in a line.
The difference is that in a permutation there are more people than there are slots in the photo.
Just as in the photo example, \emph{order matters}.
In particular, a permutation is \textbf{the number of ways to arrange $n$ objects in $k$ ordered positions}.


\subsection*{Combinations}

\section*{Classical Probability: Equally Likely Outcomes}

\end{document}


